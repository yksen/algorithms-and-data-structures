\documentclass{article}

\usepackage{amsmath}
\usepackage{enumitem}
\usepackage[polish]{babel}
\usepackage[T1]{fontenc}
\usepackage[utf8]{inputenc}
\usepackage[margin=1.5in]{geometry}

\begin{document}

\title{Algorytmy i struktury danych}
\author{}
\date{}
\maketitle

\section*{Lista zadań 1}

\subsection*{Zadanie 1}
Wyraź w notacji $O$ złożoność następujących procedur:
\begin{enumerate}[label=(\alph*)]
    \item Sprawdzenie czy liczba $n$ ma dzielnik większy od $1$ ale mniejszy od $n$.
          \begin{gather*}
              f < g \cdot c \\
              n-2 < n \cdot c \\
              n-2 < n \cdot 1 \text{ dla }n \geq 0 \\
              n-2 = O(n)
          \end{gather*}
    \item Sprawdzenie czy liczba $n$ ma dzielnik większy od $1$ ale mniejszy lub równy $\sqrt{n}$.
          \begin{gather*}
              \left \lceil {\sqrt{n}} \right \rceil - 1 = O(\sqrt{n})
          \end{gather*}
    \item Wyznaczenie wszystkich liczb pierwszych z przedziału $1..n$ algorytmem Erastotenesa
          (podziel wynik przez $n$). Skorzystaj z faktu, że
          \begin{gather*}
              \sum_{k=1}^{n} \frac{1}{k} < \ln n + 1 \\
              \frac{n}{2}-1 + \frac{n}{3}-1 + \dots + \frac{n}{k}-1 \\
              \frac{n}{2} + \frac{n}{3} + \dots + \frac{n}{k} - k \\
              n\left(\frac{1}{2} + \frac{1}{3} + \dots + \frac{1}{k}\right) - k < n \ln n - k \\
              f(n)=\frac{n \ln k - k}{n} = \ln k - \frac{k}{n} < \ln n = O(\ln n)
          \end{gather*}
    \item Wyznacz numeryczne proporcje otrzymanych wyników $(a)/(b)$ i $(b)/(c)$ dla $n = 10^6$.
          \begin{gather*}
              (a)/(b) = n/\sqrt{n} = 10^6/\left(10^6\right)^\frac{1}{2} = 10^6/10^3 = 10^3 = 1000 \\
              (b)/(c) = \sqrt{n}/\ln n = \left(10^6\right)^\frac{1}{2}/\ln 10^6 = 10^3/\ln 10^6 = 10^3/13.8 \approx 72.5
          \end{gather*}
\end{enumerate}

\subsection*{Zadanie 2}
Wyraź w notacji $O$ ile dodawań wykonasz wyznaczając n początkowych liczb Fibonacciego:
\begin{enumerate}[label=(\alph*)]
    \item Za pomocą procedury typowej rekurencyjnej.
          \begin{gather*}
              O(Fib(0) + Fib(1) + \dots + Fib(n)) \\
              Fib(n) \approx \frac{1}{\sqrt{5}}\left(\frac{1-\sqrt{5}}{2}\right)^n \\
              O\left(\left(\frac{1+\sqrt{5}}{2}\right)^n\right)
          \end{gather*}
    \item Za pomocą procedury iteracyjnej wywoływanej dla każdej liczby osobno.
          \begin{gather*}
              1 + 2 + \dots + n = \sum_{k=1}^{n}k = \frac{(a_1+a_n)n-1}{2} = \frac{(1+n-1)(n-1)}{2} = \frac{n^2-n}{2} = O(n^2)
          \end{gather*}
    \item Za pomocą procedury iteracyjnej wyznaczającej wszystkie liczby w jednej pętli.
          \begin{gather*}
              n - 1 = O(n)
          \end{gather*}
\end{enumerate}

\subsection*{Zadanie 5}
Schemat Hoernera: Przed odpowiednie wyłączenia $x$ przed nawias, pokaż, że wystarczy
dokładnie n mnożeń, aby wyliczyć wartość wielomianu stopnia $n$?
\begin{gather*}
    W(x) = a_0 + a_1x + a_2x^2 + \dots + a_nx^n
\end{gather*}
Wskazówka: zadanie wykonaj kolejno dla $n = 0, 1, 2, 3, 4$ a potem uogólnij pisząc odpowiedni algorytm
\begin{align*}
    W(x) & = a_0                                                    \\
    W(x) & = a_0 + x(a_1)                                           \\
    W(x) & = a_0 + x(a_1 + x(a_2))                                  \\
    W(x) & = a_0 + x(a_1 + x(a_2 + x(a_3)))                         \\
    W(x) & = a_0 + x(a_1 + x(a_2 + \dots + x(a_{n-1} + xa_n)))
\end{align*}

\end{document}