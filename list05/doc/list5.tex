\documentclass{article}

\usepackage{amsmath}
\usepackage{amssymb}
\usepackage{enumitem}
\usepackage[polish]{babel}
\usepackage[T1]{fontenc}
\usepackage[utf8]{inputenc}
\usepackage[margin=0.8in]{geometry}
\usepackage[edges]{forest}
\usepackage{listings}
\usepackage{multicol}

\begin{document}

\title{Algorytmy i struktury danych}
\author{}
\date{}
\maketitle

\section*{Lista zadań 5}

\subsection*{Zadanie 1}
Udowodnij, że jeśli dla pewnego ustalonego $q$, takiego że $\frac{1}{2} < q < 1$,
podczas sortowania szybkiego, procedura partition, na każdym poziomie rekurencji podzieli elementy
tablicy w stosunku $q : (1 - q)$ to algorytm wykona się w czasie $O(n log n)$. Wskazówka: udowodnij, że głębokość
rekurencji nie przekroczy - $\log n / \log q$ i zaniedbaj błędy zaokrągleń do wartości całkowitych.
\begin{center}
    \begin{forest}
        for tree={math content, minimum size=3ex, inner sep=1pt, s sep=7mm, anchor=south}
        [n[(1-q)n][qn[(1-q)qn][q^2n[(1-q)q^kn, edge=dashed][q^kn, edge=dashed]]]]
    \end{forest}
\end{center}
\begin{gather*}
    q^kn = 1 \\
    q^k = \frac{1}{n} \\
    \log q^k = \log n^{-1} \\
    k \log q = -\log n \\
    k = -\frac{\log n}{\log q}
\end{gather*}

\subsection*{Zadanie 2}
Ile porównań (zapisz wyniki w notacji $O$) wykona algorytm \verb+quicksort+ z
procedurą \verb+partition+ w wersji Hoare'a, a ile w wersji z procedurę \verb+partition+ w wersji Lomuto dla
danych: (a) posortowanych rosnąco, (b) posortowanych malejąco, (c) o identycznych kluczach?

\subsection*{Zadanie 3}
Napisz wzór na numer kubełka, do którego należy wrzucić liczbę $x$ w sortowaniu kubełkowym, jeśli
kubełków jest $n$, a elementy tablicy mieszczą się przedziale $(a, b)$. Numeracja zaczyna się od $0$.
\begin{gather*}
    k = \left \lfloor \frac{n}{b - a} \cdot \left(x - a\right) \right \rfloor \\
    a \leq x < b \\
    0 \leq x - a < b - a \\
    0 \leq \frac{x-a}{b-a} < 1 \\
    \left \lfloor \frac{x-a}{b-a} \cdot n \right \rfloor
\end{gather*}

\subsection*{Zadanie 4}
Dla jakich danych sortowanie metodą kubełkową ma złożoność $O(n^2)$?
\begin{center}
    Dla danych rozłożonych bardzo nierównomiernie których większość trafi do jednego kubełka, gdyż wywołany będzie na nich
    algorytm sortowania przez wstawianie o złożoności $O(n^2)$.
\end{center}

\subsection*{Zadanie 5}
Jak obliczyć $k$-tą od końca cyfrę w liczby $x$? Jak obliczyć ilość cyfr liczby $x$? Przyjmujemy
układ dziesiętny. Jak wyniki zmienią się w układzie pozycyjnym gdzie różnych cyfr jest $m$ a ich
wartości $x$ należą do przedziału $0 \leq x < m$?
\begin{align*}
    x_k & = \left \lfloor \frac{x}{10^k} \right \rfloor \mod 10 & n & = \left \lceil \log_{10} x \right \rceil + 1 \\
    x_k & = \left \lfloor \frac{x}{m^k} \right \rfloor \mod m   & n & = \left \lceil \log_m x \right \rceil + 1
\end{align*}

\pagebreak
\subsection*{Zadanie 6}
Posortuj metodą sortowania pozycyjnego liczby: 101, 345, 103, 333, 432, 132, 543, 651,
791, 532, 987, 910, 643, 641, 12, 342, 498, 987, 965, 322, 121, 431, 350. W pisemnym
rozwiązaniu pokaż, jak wygląda zawartość kolejek, za każdym razem, gdy tablica
wyjściowa jest pusta i wszystkie liczby znajdują się w kolejkach, oraz jak wygląda
tablica wyjściowa, za każdym razem, gdy sortowanie ze względu na kolejną cyfrę jest już zakończone.

\begin{align*}
    0: & \left(910, 350\right)                     \\
    1: & \left(101, 651, 791, 641, 121, 431\right) \\
    2: & \left(432, 132, 532, 12, 342, 322\right)  \\
    3: & \left(103, 333, 543, 643\right)           \\
    4: & \left(\right)                             \\
    5: & \left(345, 965\right)                     \\
    6: & \left(\right)                             \\
    7: & \left(987, 987\right)                     \\
    8: & \left(498\right)                          \\
    9: & \left(\right)
\end{align*}
(910, 350, 101, 651, 791, 641, 121, 431, 432, 132, 532, 12, 342, 322, 103, 333, 543, 643, 345, 965, 987, 987, 498)

\begin{align*}
    0: & \left(101, 103\right)                \\
    1: & \left(910, 12\right)                 \\
    2: & \left(121, 322\right)                \\
    3: & \left(431, 432, 132, 532, 333\right) \\
    4: & \left(641, 342, 543, 643, 345\right) \\
    5: & \left(350, 651\right)                \\
    6: & \left(965\right)                     \\
    7: & \left(\right)                        \\
    8: & \left(987, 987\right)                \\
    9: & \left(791, 498\right)
\end{align*}
(101, 103, 910, 12, 121, 322, 431, 432, 132, 532, 333, 641, 342, 543, 643, 345, 350, 651, 965, 987, 987, 791, 498)

\begin{align*}
    0: & \left(12\right)                      \\
    1: & \left(101, 103, 121, 132\right)      \\
    2: & \left(\right)                        \\
    3: & \left(322, 333, 342, 345, 350\right) \\
    4: & \left(431, 432, 498\right)           \\
    5: & \left(532, 543\right)                \\
    6: & \left(641, 643, 651\right)           \\
    7: & \left(791\right)                     \\
    8: & \left(\right)                        \\
    9: & \left(910, 965, 987, 987\right)
\end{align*}
(12, 101, 103, 121, 132, 322, 333, 342, 345, 350, 431, 432, 498, 532, 543, 641, 643, 651, 791, 910, 965, 987, 987)

\pagebreak
\subsection*{Zadanie 7}
Które z procedur sortujących:
\begin{enumerate}[label=(\alph*)]
    \item insertionSort (przez wstawianie), \\
          jest stabilny, gdyż każdy kolejny, zaczynając od początku tablicy, sortowany element jest wstawiany na koniec
          posortowanej części tablicy, więc w przypadku elementów o tej samej wartości, zachowana jest kolejność
    \item quickSort (szybkie), \\
          nie jest stabilny, przykładowo dla tablicy \verb+{2, 1, 1}+ już po pierwszym wywołaniu \verb+partition+
          tracimy względna kolejność elementów o tej samej wartości, a kolejne wywołanie nie ma już nawet szansy jej
          przywrócić bo obejmie ono elementy o indeksach większych od $0$
    \item heapSort (przez kopcowanie), \\
          nie jest stabilny, przykładowo dla tablicy \verb+{1, 1}+ sama budowa kopca nie dokona żadnych zamian
          elementów, ale już po pierwszym wyciągnięciu z kopca elementu o największej wartości czyli z indeksu zerowego
          tablicy tracimy względna kolejność elementów o tej samej wartości, gdyż jedynki zostaną wyjęte z kopca
          w tej samej kolejności w której były w tablicy ale trafiać będą na jej koniec przez co zamienią się miejscami
    \item mergeSort (przez złączanie), \\
          jest stabilny, procedura \verb+merge+ zachowuje względną kolejność elementów o tej samej wartości gdyż w
          przypadku porównywania dwóch elementów o tej samej wartości, wstawiany jest ten, który jest w tablicy
          pierwotnej na mniejszym indeksie
    \item countingSort (przez zliczanie), \\
          jest stabilny, ponieważ do tablicy wynikowej elementy są wstawiane po kolei z tablicy pierwotnej iterowanej
          od końca na kolejne dekrementowane indeksy na podstawie liczników
    \item radixSort (pozycyjne), \\
          jest stabilny, dzięki temu, że elementy umieszczana są kolejno w kolejkach FIFO z których później są wyciągane
          do tablicy pierwotnej
    \item bucketSort (kubełkowe) \\
          jest stabilny, ponieważ jest on połączeniem sortowania przez zliczanie i sortowania przez wstawianie, które
          oba są stabilne
\end{enumerate}
są stabilne? W każdym przypadku uzasadnij stabilność lub znajdź konkretny przykład
danych, dla których algorytm nie zachowa się stabilnie.

\subsection*{Zadanie 8}
Napisz funkcję void \verb+counting_sort(node* lista, int m);+ sortującą przez zliczanie
listę linkowaną liczb całkowitych nieujemnych mniejszych od $m$. Procedura nie powinna
usuwać ani tworzyć nowych węzłów, tylko sprytnie zmieniać pola next wykorzystując
tylko $O(m)$ dodatkowej pamięci na wskaźniki.

\subsection*{Zadanie 9}
(algorytm Hoare'a) Korzystając funkcji \verb+int partition(int t[], int n)+ znanej z
algorytmu sortowania szybkiego napisz funkcję \verb+int kty(int t[], int n)+, której wynikiem
będzie $k$-ty co do wielkości element początkowo nieposortowanej tablicy \verb+t+. Średnia
złożoność Twojego algorytmu powinna wynieść $O(n)$.

\end{document}