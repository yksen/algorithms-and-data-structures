\documentclass{article}

\usepackage{amsmath}
\usepackage{amssymb}
\usepackage{enumitem}
\usepackage[polish]{babel}
\usepackage[T1]{fontenc}
\usepackage[utf8]{inputenc}
\usepackage[margin=0.8in]{geometry}
\usepackage[edges]{forest}
\usepackage{listings}
\usepackage{multicol}

\begin{document}

\title{Algorytmy i struktury danych}
\author{}
\date{}
\maketitle

\section*{Lista zadań 6}

\subsection*{Zadanie 1}
Jakie informacje przechowujemy w węźle drzewa czerwono-czarnego?
Zadeklaruj strukturę \verb+RBTnode+ tak, by dziedziczyła z \verb+BSTnode+. Podaj definicję drzewa czerwono czarnego.
\begin{lstlisting}
    struct NodeBST
    {
        int32_t value;
        NodeBST* left;
        NodeBST* right;  
    };

    struct NodeRBT : public NodeBST
    {
        bool isBlack;
    };
\end{lstlisting}
Drzewo czerwono-czarne musi przestrzegać następujące wymagania:
\begin{enumerate}
    \item Każdy węzeł jest albo czerwony albo czarny.
    \item Korzeń jest czarny.
    \item Każdy liść jest czarny.
    \item Czerwony węzeł ma czarnego ojca.
    \item Każda ścieżka od korzenia do liścia ma tę samą liczbę czarnych węzłów.
\end{enumerate}

\subsection*{Zadanie 2}
\begin{enumerate}[label=(\alph*)]
    \item Jaka może być minimalna, a jaka maksymalna ilość kluczy w drzewie czerwono-czarnym
          o ustalonej czarnej wysokości równej $h_B$?
          \begin{gather*}
              \text{min} = 2^{h_B} - 1 \\
              \text{max} = 2^{2(h_B)} - 1
          \end{gather*}
    \item Znajdź maksymalną i minimalną wartość stosunku ilości węzłów czerwonych do
          czarnych w drzewie czerwono-czarnym.
\end{enumerate}
\begin{multicols*}{2}
    \begin{center}
        min = $0/1$ \\[1ex]
        \begin{forest}
            for tree={circle, minimum size=3ex, inner sep=1pt, s sep=7mm, anchor=south, fill=black, text=white},
            []
        \end{forest}
    \end{center}
    \columnbreak
    \begin{center}
        max = $2/1$ \\[1ex]
        \begin{forest}
            for tree={circle, minimum size=3ex, inner sep=1pt, s sep=7mm, anchor=south, fill=black, text=white},
            [[, fill=red][, fill=red]]
        \end{forest}
    \end{center}
\end{multicols*}

\subsection*{Zadanie 3}
Uzasadnij posługując się rysunkiem i opisem, że operacje wykonywane w trakcie wstawiania
do drzewa czerwono-czarnego (rotacja i przekolorowanie) nie zmieniają ilości
czarnych węzłów, na żadnej ścieżce od korzenia do liścia. \\

\noindent
Operacja przekolorowania nie zmienia ilości czarnych węzłów na żadnej ścieżce od korzenia do liścia, ponieważ
czarny kolor z węzła położonego bliżej korzenia czy też rodzica zostanie przetransferowany do każdego z jego dzieci
których ścieżka do korzenia przechodzi przez niego, a więc czarna wysokość nie ulegnie zmianie na żadnej ścieżce. \\
\begin{center}
    \begin{forest}
        for tree={circle, minimum size=3ex, inner sep=1pt, s sep=7mm, anchor=south, fill=black, text=white},
        [B[A, fill=red][C, fill=red]]
    \end{forest}
    \begin{forest}
        for tree={circle, minimum size=3ex, inner sep=1pt, s sep=7mm, anchor=south, fill=black, text=white},
        [B, fill=red[A][C]]
    \end{forest}
\end{center}
\noindent
W przypadku rotacji których wyróżniamy dwa rodzaje, tylko przy jednej z nich dochodzi do zmiany kolorów węzłów i w tej
sytuacji węzły które w jej trakcie zmieniają swój poziom w drzewie jednocześnie zamieniają sie kolorami, co efektywnie
skutkuje rezultatem analogicznym do przekolorowania, czyli czarna wysokość na żadnej ścieżce od korzenia do
liścia nie ulega zmianie. \\
\begin{center}
    \begin{forest}
        for tree={circle, minimum size=3ex, inner sep=1pt, s sep=7mm, anchor=south, fill=black, text=white},
        [C, fill=red[A[, phantom][B]][, phantom]]
    \end{forest}
    \begin{forest}
        for tree={circle, minimum size=3ex, inner sep=1pt, s sep=7mm, anchor=south, fill=black, text=white},
        [C, fill=red[B[A][, phantom]][, phantom]]
    \end{forest}
    \begin{forest}
        for tree={circle, minimum size=3ex, inner sep=1pt, s sep=7mm, anchor=south, fill=black, text=white},
        [B, fill=red[A[, phantom][, phantom]][C[, phantom][, phantom]]]
    \end{forest}
\end{center}

\subsection*{Zadanie 4}
\begin{enumerate}[label=(\alph*)]
    \item Narysuj poprawne drzewo czerwono-czarne w którym na lewo od korzenia jest $1$
          węzeł a na prawo $7$ węzłów.
          \begin{center}
              \begin{forest}
                  for tree={circle, minimum size=3ex, inner sep=1pt, s sep=7mm, anchor=south, fill=black, text=white},
                  [[][, fill=red[[, fill=red][, fill=red]][[, fill=red][, fill=red]]]]
              \end{forest}
          \end{center}
    \item Czy istnieje poprawne drzewo czerwono-czarne, w którym na lewo od korzenia będzie
          $100$ razy mniej węzłów niż na prawo od korzenia?
          \begin{gather*}
              100 \left(2^{h_B-1}-1\right) < 2^{2(h_B-1)}-1
          \end{gather*}
          \begin{center}
              \begin{tabular}{c c c c}
                  $h_B-1$ & \textbf{min} & $100$ \textbf{min} & \textbf{max} \\
                  \hline
                  $1$     & $0$          & $0$                & $0$          \\
                  $2$     & $1$          & $100$              & $3$          \\
                  $3$     & $3$          & $300$              & $15$         \\
                  $4$     & $7$          & $700$              & $63$         \\
                  $5$     & $15$         & $1500$             & $255$        \\
                  $6$     & $31$         & $3100$             & $1023$       \\
                  $7$     & $63$         & $6300$             & $4095$       \\
                  $8$     & $127$        & $12700$            & $16383$      \\
              \end{tabular}
          \end{center}
          Istnieje takie drzewo czerwono-czarne, w którym na lewo od korzenia jest $100$ razy mniej węzłów niż na prawo.
          Dla czarnej wysokości $h_B=9$ po lewej stronie minimalnie może być $127$ węzłów, a po prawej $16383$.
          \begin{align*}
              100 \left(2^{8-1}-1\right) & < 2^{2(8-1)}-1 \\
              100 \left(2^{7}-1\right)   & < 2^{2(7)}-1   \\
              100 \left(127\right)       & < 2^{14}-1     \\
              12700                      & < 16383
          \end{align*}
\end{enumerate}

\pagebreak
\subsection*{Zadanie 5}
W poniższym drzewie czerwono-czarnym:
\begin{center}
    \begin{forest}
        for tree={circle, minimum size=3ex, inner sep=1pt, s sep=7mm, anchor=south, fill=black, text=white},
        [5[3, fill=red[1][4]][11[9, fill=red][, phantom]]]
    \end{forest}
\end{center}
\begin{enumerate}[label=-]
    \item wstaw do niego \verb+10+.
          \begin{center}
              \begin{forest}
                  [, phantom, for tree={circle, minimum size=3ex, inner sep=1pt, s sep=5mm, anchor=south, fill=black, text=white},
                  [5[3, fill=red[1][4]][11[9, fill=red[, phantom][10, fill=red]][, phantom]]]
                  [5[3, fill=red[1][4]][11[10, fill=red[9, fill=red][, phantom]][, phantom]]]
                  [5[3, fill=red[1][4]][10, fill=red[9, fill=red][11]]]
                  [5[3, fill=red[1][4]][10[9, fill=red][11, fill=red]]]
                  ]
              \end{forest}
          \end{center}
    \item usuń z wyjściowego drzewa \verb+1+.
          \begin{center}
              \begin{forest}
                  [, phantom, for tree={circle, minimum size=3ex, inner sep=1pt, s sep=5mm, anchor=south, fill=black, text=white},
                  [5[3, fill=red[, phantom][4]][10[9, fill=red][11, fill=red]]]
                  [5[3[, phantom][4, fill=red]][10[9, fill=red][11, fill=red]]]
                  ]
              \end{forest}
          \end{center}
\end{enumerate}

\pagebreak
\subsection*{Zadanie 6}
(3 pkt.) Do pustego drzewa czerwono-czarnego wstaw kolejno $20$ przypadkowych kluczy.
Następnie usuń je w tej samej kolejności w jakiej wstawiałeś. Przypadkowymi kluczami
są kolejne litery Twojego nazwiska, imienia i adresu. Zadanie wykonujemy na kartce
(lub w pliku) i oddajemy prowadzącemu. Zadanie jest obowiązkowe. \\
\begin{center}
    \textbf{Wstawianie} \\[1em]
    \begin{forest}
        [, phantom, for tree={circle, minimum size=3ex, inner sep=1pt, s sep=5mm, l sep=0mm, l=0mm, anchor=south, fill=black, text=white},
        [K[, phantom][, phantom]]
        [K[A, fill=red][, phantom]]
        [K[A, fill=red][M, fill=red]]
        [K[A[, phantom][I, fill=red]][M]]
        [K[A[, phantom][I, fill=red]][M[L, fill=red][, phantom]]]
        [K[A[, phantom][I, fill=red]][M[L, fill=red][S, fill=red]]]
        ]
    \end{forest}
\end{center}
\begin{center}
    \begin{forest}
        [, phantom, for tree={circle, minimum size=3ex, inner sep=1pt, s sep=5mm, l sep=0mm, l=0mm, anchor=south, fill=black, text=white},
        [K[E[A, fill=red][I, fill=red]][M[L, fill=red][S, fill=red]]]
        [K[E[A, fill=red][I, fill=red]][M, fill=red[L[, phantom][L, fill=red]][S]]]
        [K[E, fill=red[A][I[E, fill=red][, phantom]]][M, fill=red[L[, phantom][L, fill=red]][S]]]
        [K[E, fill=red[A][G[E, fill=red][I, fill=red]]][M, fill=red[L[, phantom][L, fill=red]][S]]]
        ]
    \end{forest}
\end{center}
\begin{center}
    \begin{forest}
        [, phantom, for tree={circle, minimum size=3ex, inner sep=1pt, s sep=5mm, l sep=0mm, l=0mm, anchor=south, fill=black, text=white},
        [K[E, fill=red[A[, phantom][A, fill=red]][G[E, fill=red][I, fill=red]]][M, fill=red[L[, phantom][L, fill=red]][S]]]
        [K[E, fill=red[A[, phantom][A, fill=red]][G[E, fill=red][I, fill=red]]][M, fill=red[L[, phantom][L, fill=red]][S[P, fill=red][, phantom]]]]
        ]
    \end{forest}
\end{center}
\begin{center}
    \begin{forest}
        [, phantom, for tree={circle, minimum size=3ex, inner sep=1pt, s sep=5mm, l sep=0mm, l=0mm, anchor=south, fill=black, text=white},
        [K[E, fill=red[A[, phantom][A, fill=red]][G[E, fill=red][I, fill=red]]][M, fill=red[L[, phantom][L, fill=red]][P[O, fill=red][S, fill=red]]]]
        [K[E, fill=red[A[, phantom][A, fill=red]][G[E, fill=red][I, fill=red]]][M, fill=red[L[L, fill=red][L, fill=red]][P[O, fill=red][S, fill=red]]]]
        ]
    \end{forest}
\end{center}
\begin{center}
    \begin{forest}
        [, phantom, for tree={circle, minimum size=3ex, inner sep=1pt, s sep=5mm, l sep=0mm, l=0mm, anchor=south, fill=black, text=white},
        [K[E[A[, phantom][A, fill=red]][G[E, fill=red][I, fill=red]]][M[L[L, fill=red][L, fill=red]][P, fill=red[O][S[, phantom][S, fill=red]]]]]
        [K[E[A[, phantom][A, fill=red]][G[E, fill=red][I, fill=red]]][M[L, fill=red[L[K, fill=red][, phantom]][L]][P, fill=red[O][S[, phantom][S, fill=red]]]]]
        ]
    \end{forest}
\end{center}
\begin{center}
    \begin{forest}
        [, phantom, for tree={circle, minimum size=3ex, inner sep=1pt, s sep=5mm, l sep=0mm, l=0mm, anchor=south, fill=black, text=white},
        [K[E[A[A, fill=red][A, fill=red]][G[E, fill=red][I, fill=red]]][M[L, fill=red[L[K, fill=red][, phantom]][L]][P, fill=red[O][S[, phantom][S, fill=red]]]]]
        [K[E[A[A, fill=red][A, fill=red]][G[E, fill=red][I, fill=red]]][M[L, fill=red[L[K, fill=red][, phantom]][L]][P, fill=red[O][S[S, fill=red][W, fill=red]]]]]
        ]
    \end{forest}
\end{center}
\begin{center}
    \begin{forest}
        [, phantom, for tree={circle, minimum size=3ex, inner sep=1pt, s sep=5mm, l sep=0mm, l=0mm, anchor=south, fill=black, text=white},
        [K[E[A[A, fill=red][A, fill=red]][G[E, fill=red][I, fill=red]]][M, fill=red[L[L[K, fill=red][, phantom]][L]][P[O][S, fill=red[S[R, fill=red][, phantom]][W]]]]]
        [K[E[A[A, fill=red][A, fill=red]][G[E, fill=red][I, fill=red]]][M, fill=red[L[L[K, fill=red][, phantom]][L]][P[O[, phantom][O, fill=red]][S, fill=red[S[R, fill=red][, phantom]][W]]]]]
        ]
    \end{forest}
\end{center}

\pagebreak
\begin{center}
    \textbf{Usuwanie} \\[1em]
    \begin{forest}
        [, phantom, for tree={circle, minimum size=3ex, inner sep=1pt, s sep=5mm, l sep=0mm, l=0mm, anchor=south, fill=black, text=white},
        [K[E[A[A, fill=red][A, fill=red]][G[E, fill=red][I, fill=red]]][M, fill=red[L[L][L]][P[O[, phantom][O, fill=red]][S, fill=red[S[R, fill=red][, phantom]][W]]]]]
        [K[E[A[A, fill=red][, phantom]][G[E, fill=red][I, fill=red]]][M, fill=red[L[L][L]][P[O[, phantom][O, fill=red]][S, fill=red[S[R, fill=red][, phantom]][W]]]]]
        ]
    \end{forest}
\end{center}
\begin{center}
    \begin{forest}
        [, phantom, for tree={circle, minimum size=3ex, inner sep=1pt, s sep=5mm, l sep=0mm, l=0mm, anchor=south, fill=black, text=white},
        [K[E[A[A, fill=red][, phantom]][G[E, fill=red][I, fill=red]]][O, fill=red[L[L][L]][P[O][S, fill=red[S[R, fill=red][, phantom]][W]]]]]
        [K[E[A[A, fill=red][, phantom]][G[E, fill=red][, phantom]]][O, fill=red[L[L][L]][P[O][S, fill=red[S[R, fill=red][, phantom]][W]]]]]
        ]
    \end{forest}
\end{center}
\begin{center}
    \begin{forest}
        [, phantom, for tree={circle, minimum size=3ex, inner sep=1pt, s sep=5mm, l sep=0mm, l=0mm, anchor=south, fill=black, text=white},
        [K[E[A[A, fill=red][, phantom]][G[E, fill=red][, phantom]]][P, fill=red[O[L[L, fill=red][, phantom]][O]][S[S[R, fill=red][, phantom]][W]]]]
        [K[E[A[A, fill=red][, phantom]][G[E, fill=red][, phantom]]][P, fill=red[O[L[L, fill=red][, phantom]][O]][S[R][W]]]]
        ]
    \end{forest}
\end{center}
\begin{center}
    \begin{forest}
        [, phantom, for tree={circle, minimum size=3ex, inner sep=1pt, s sep=5mm, l sep=0mm, l=0mm, anchor=south, fill=black, text=white},
        [K[E[A[A, fill=red][, phantom]][G]][P, fill=red[O[L[L, fill=red][, phantom]][O]][S[R][W]]]]
        [K[E[A[A, fill=red][, phantom]][G]][P, fill=red[O[L][O]][S[R][W]]]]
        [K[A[A][G]][P, fill=red[O[L][O]][S[R][W]]]]
        ]
    \end{forest}
\end{center}
\begin{center}
    \begin{forest}
        [, phantom, for tree={circle, minimum size=3ex, inner sep=1pt, s sep=5mm, l sep=0mm, l=0mm, anchor=south, fill=black, text=white},
        [P[K[A[A, fill=red][, phantom]][O, fill=red[L][O]]][S[R][W]]]
        [P[K[A][O, fill=red[L][O]]][S[R][W]]]
        [O[K[A][L]][R[O][S[, phantom][W, fill=red]]]]
        [O[K[A][L]][S[R][W]]]
        ]
    \end{forest}
\end{center}
\begin{center}
    \begin{forest}
        [, phantom, for tree={circle, minimum size=3ex, inner sep=1pt, s sep=5mm, l sep=0mm, l=0mm, anchor=south, fill=black, text=white},
        [O[K[A, fill=red][, phantom]][S, fill=red[R][W]]]
        [O[K[A, fill=red][, phantom]][W[R, fill=red][, phantom]]]
        [O[A][W[R, fill=red][, phantom]]]
        [R[O][W]]
        [R[O, fill=red][, phantom]]
        [O[, phantom][, phantom]]
        [n, fill=green[, phantom][, phantom]]
        ]
    \end{forest}
\end{center}

\pagebreak
\subsection*{Zadanie 7}
Analizując kod programu \verb+RBT.h+ udowodnij, że w trakcie wstawiania do drzewa czerwono-czarnego
wykonają się co najwyżej dwie rotacje. Czy tak samo jest w przypadku usuwania?
\begin{multicols}{2}
    \noindent
    Bezpośrednio w samej procedurze \verb+insert+ nie ma żadnych rotacji. Rotacje mogą wystąpić dopiero w procedurze
    \verb+fix_up+, która jest wywoływana jest co najwyżej raz, zaraz po wstawieniu nowego węzła.
    \begin{lstlisting}
...
for (;;)
    if (x < t->key)
    {
        if (t->left)
            t = t->left;
        else
        {
            t->left = new node(x, t);
            fix_up(t->left);
            break;
        }
    }
    else
    {
        if (t->right)
            t = t->right;
        else
        {
            t->right = new node(x, t);
            fix_up(t->right);
            break;
        }
    }
...
    \end{lstlisting}
    \columnbreak
    Wiedząc że procedura \verb+fix_up+ wykonuje się co najwyżej jednokrotnie, możemy zatem stwierdzić, że maksymalna
    liczba rotacji wynosi dwie, co uwidaczniają poniższe fragmenty kodu z procedury \verb+fix_up+.
    \begin{lstlisting}
...
else
{
    if (t == p->right)
        rotate_left(p);
    rotate_right(pp);
    return;
}
...
else
{

    if (t == p->left)
        rotate_right(p);
    rotate_left(pp);
    return;
}
...
    \end{lstlisting}
\end{multicols}
\noindent
W przypadku usuwania, istnieje możliwość wystąpienia więcej niż dwóch rotacji, dlatego że procedura \verb+rem_fix_up+
wywołana najwyżej raz zawiera pętlę \verb+while+ której ilość iteracji zależna jest od wysokości drzewa, a co za tym
idzie dla większych wysokości drzewa, ilość możliwych rotacji będzie wzrastać.

\pagebreak
\subsection*{Zadanie 8}
Uzasadnij, że rozmiar stosu $(n = 100)$ przyjęty w procedurach \verb+insert+ i \verb+remove+ w pliku
\verb+RBnpnr.h+ nigdy nie okaże się za mały. \\

\noindent
Wiedząc że na stos przyjęty w procedurach procedurach \verb+insert+ i \verb+remove+ trafiają kolejne odwiedzone węzły
drzewa w trakcie szukania miejsca do wstawienia nowego węzła w przypadku procedury \verb+insert+ oraz dodatkowo
w trakcie szukania następnika w przypadku procedury \verb+remove+ wiadomo że nie trafi tam więcej węzłów niż wynosi
wysokość samego drzewa. Przyglądając się ilości węzłów w drzewie o takiej wysokości
\begin{gather*}
    n_{min} = 2^{h-1} = n_{max} = 2^{99} - 1 = 633825300114114700748351602687 \\
    n_{max} = 2^h - 1 = n_{max} = 2^{100} - 1 = 1267650600228229401496703205375
\end{gather*}
możemy zauważyć, że stos nigdy nie będzie za mały, ponieważ gdyby nawet przyjąć że każdy węzeł trzyma jeden bit
informacji, to łączna pamięć tego drzewa wynosząca $7.923 \cdot 10^{28}$ bajtów, lub inaczej $79.23$ ronnabajtów jest
praktycznie nieosiągalna dla współczesnych komputerów. Dla porównania taka pamięć byłaby w stanie pomieścić
$8.7 \cdot 10^{11}$ razy szacowany rozmiar deep webu, który wynosi około $91000$ terabajtów, czy też $4.7 \cdot 10^{14}$
raza szacowany rozmiar widzialnej sieci internetowej, który wynosi około $170$ terabajtów.

\end{document}