\documentclass{article}

\usepackage{amsmath}
\usepackage{enumitem}
\usepackage[polish]{babel}
\usepackage[T1]{fontenc}
\usepackage[utf8]{inputenc}
\usepackage[margin=1.5in]{geometry}

\begin{document}

\title{Algorytmy i struktury danych}
\author{}
\date{}
\maketitle

\section*{Lista zadań 1}
\subsection*{Zadanie 1}
Wyraź w notacji $O$ złożoność następujących procedur:
\begin{enumerate}[label=(\alph*)]
    \item Sprawdzenie czy liczba $n$ ma dzielnik większy od $1$ ale mniejszy od $n$.
          \begin{gather*}
              f < g \cdot c \\
              n-2 < n \cdot c \\
              n-2 < n \cdot 1 \text{ dla }n \geq 0 \\
              n-2 = O(n)
          \end{gather*}
    \item Sprawdzenie czy liczba $n$ ma dzielnik większy od $1$ ale mniejszy lub równy $\sqrt{n}$.
          \begin{gather*}
              \left \lceil {\sqrt{n}} \right \rceil - 1 = O(\sqrt{n})
          \end{gather*}
    \item Wyznaczenie wszystkich liczb pierwszych z przedziału $1..n$ algorytmem Erastotenesa
          (podziel wynik przez $n$). Skorzystaj z faktu, że
          \begin{gather*}
              \sum_{k=1}^{n} \frac{1}{k} < \ln n + 1 \\
              \frac{n}{2}-1 + \frac{n}{3}-1 + \dots + \frac{n}{k}-1 \\
              \frac{n}{2} + \frac{n}{3} + \dots + \frac{n}{k} - k \\
              n(\frac{1}{2} + \frac{1}{3} + \dots + \frac{1}{k}) - k < n \ln n - k \\
              f(n)=\frac{n \ln n - k}{n} = \ln n - \frac{k}{n} < \ln n = O(\ln n)
          \end{gather*}
    \item Wyznacz numeryczne proporcje otrzymanych wyników $(a)/(b)$ i $(b)/(c)$ dla $n = 10^6$.
          \begin{gather*}
              (a)/(b) = n/\sqrt{n} = 10^6/(10^6)^\frac{1}{2} = 10^6/10^3 = 10^3 = 1000 \\
              (b)/(c) = \sqrt{n}/\ln n = (10^6)^\frac{1}{2}/\ln 10^6 = 10^3/\ln 10^6 = 10^3/13.8 \approx 72.5
          \end{gather*}

\end{enumerate}

\end{document}