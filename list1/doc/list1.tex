\documentclass{article}

\usepackage{amsmath}
\usepackage{enumitem}
\usepackage[polish]{babel}
\usepackage[T1]{fontenc}
\usepackage[utf8]{inputenc}
\usepackage[margin=1.5in]{geometry}

\begin{document}

\title{Algorytmy i struktury danych}
\author{}
\date{}
\maketitle

\section*{Lista zadań 1}
\subsection*{Zadanie 1}
Wyraź w notacji $O$ złożoność następujących procedur:
\begin{enumerate}[label=(\alph*)]
    \item Sprawdzenie czy liczba $n$ ma dzielnik większy od $1$ ale mniejszy od $n$.
          \begin{gather*}
              O(n)
          \end{gather*}
    \item Sprawdzenie czy liczba $n$ ma dzielnik większy od $1$ ale mniejszy lub równy $\sqrt{n}$.
          \begin{gather*}
              O(\sqrt{n})
          \end{gather*}
    \item Wyznaczenie wszystkich liczb pierwszych z przedziału $1..n$ algorytmem Erastotenesa
          (podziel wynik przez $n$). Skorzystaj z faktu, że
          \begin{gather*}
              \sum_{k=1}^{n} \frac{1}{k} < \ln n + 1 \\
              n/2 + n/3 + n/4 + \dots + n/n = n \sum_{k=2}^{n} \frac{1}{k} = n \left( \ln n + 1 - 1 \right) = n \ln n \\
              O(\frac{n\ln n}{n}) \implies O(\ln n)
          \end{gather*}
    \item Wyznacz numeryczne proporcje otrzymanych wyników $(a)/(b)$ i $(b)/(c)$ dla $n = 10^6$.
          \begin{gather*}
              (a)/(b) = n/\sqrt{n} = 10^6/(10^6)^\frac{1}{2} = 10^6/10^3 = 10^3 = 1000 \\
              (b)/(c) = n/\ln n = 10^6/\ln(10^6) = 10^6/13.816 \approx 72382.414
          \end{gather*}

\end{enumerate}

\end{document}