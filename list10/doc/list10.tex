\documentclass{article}

\usepackage{amsmath}
\usepackage{amssymb}
\usepackage{enumitem}
\usepackage[polish]{babel}
\usepackage[T1]{fontenc}
\usepackage[utf8]{inputenc}
\usepackage[margin=0.8in]{geometry}
\usepackage[edges]{forest}
\usepackage{listings}
\usepackage{multicol}
\usepackage{tikz}
\usetikzlibrary{arrows.meta, matrix}
\tikzset{
    v/.style={draw, circle, node distance={20mm}, thick}, 
    e/.style={midway, sloped, above}}

\begin{document}

\title{Algorytmy i struktury danych}
\author{}
\date{}
\maketitle

\section*{Lista zadań 10}

\subsection*{Zadanie 1}
Zastosuj strukturę \verb|UnionFind3| (znajdziesz ją pliku \verb|UnionFind3.cc| w materiałach do wykładu 10)
do sprawdzenia czy w tablicy \verb|bool t[n][n]| istnieje ścieżka zawierająca same
jedynki (true): (a) od pola \verb|t[0][0]| do \verb|t[n-1][n-1]| (b) od pierwszego do ostatniego
wiersza (tzn. jakaś komórka z pierwszego wiersza jest połączona ścieżką z jakąś komórką
z ostatniego wiersza). Za ścieżkę uważamy ciąg pól tablicy, które stykają się krawędzią
(różną się o 1 numerem kolumny albo wiersza).

\subsection*{Zadanie 2}
(a) Zastosuj strukturę \verb|UnionFind| do sprawdzenia ile wysp jedynek zawiera tablica
\verb|bool t[n][n]|. Za wyspę uważamy zbiór jedynek taki, że z każdej do każdej można
przejść ścieżką zawierającą same jedynki poruszając się tylko w poziomie i pionie.
(b) ten sam problem rozwiąż za pomocą zmodyfikowanego algorytmu DFS.

\subsection*{Zadanie 3}
(3 pkt) Dla podanego w formacie macierzy sąsiedztwa grafu zasymuluj na kartce działanie
każdego z algorytmów: (a) DSF z wierzchołka A, (b) BSF z wierzchołka A, (c) MST
Kruskal, (d) MST Prim z wierzchołka A, (e) Dijsktra z wierzchołka A, (f) Dijsktra z
wierzchołka D.
\begin{center}$
        \begin{pmatrix}
              & A & B & C & D & E & F \\
            A &   & 4 &   &   & 1 & 2 \\
            B & 4 &   & 2 &   & 2 &   \\
            C &   & 2 &   & 8 &   &   \\
            D &   &   & 8 &   & 3 & 6 \\
            E & 1 & 2 &   & 3 &   & 7 \\
            F & 2 &   &   & 6 & 7 &
        \end{pmatrix}$
\end{center}
Na kartce powinien znajdować się rysunek grafu, z zaznaczonym wynikiem działania algorytmu
oraz wypunktowane kolejne kroki algorytmu np. sortuję krawędzie; sprawdzam
krawędź AB ale A i B są już w jednym zbiorze; sprawdzam sąsiada B ale jest już czarny;
sprawdzam krawędź AB, wykonuję \verb|union(A,B)| i dodaję AB do drzewa; \verb|Q.getmin()|
daje wierzchołek $C$; sprawdzam że $key(C) > key(B) + |BC|$ i robię \verb|decrease_key(C,7)|
\begin{center}
    \begin{tikzpicture}
        \node[v] (E) {E};
        \node[v] (A)[below left of=E] {A};
        \node[v] (B)[below right of=E] {B};
        \node[v] (F)[above left of=E] {F};
        \node[v] (D)[above right of=E] {D};
        \node[v] (C)[below right of=D] {C};
        \draw (A) -- node[e] {4} (B);
        \draw (A) -- node[e] {1} (E);
        \draw (A) -- node[e] {2} (F);
        \draw (B) -- node[e] {2} (C);
        \draw (B) -- node[e] {2} (E);
        \draw (C) -- node[e] {8} (D);
        \draw (D) -- node[e] {3} (E);
        \draw (D) -- node[e] {6} (F);
        \draw (E) -- node[e] {7} (F);
    \end{tikzpicture}
\end{center}
\begin{enumerate}[label=(\alph*)]
    \item
          \begin{center}
              \begin{tikzpicture}
                  \node[v] (E) {E};
                  \node[v] (A)[below left of=E] {A};
                  \node[v] (B)[below right of=E] {B};
                  \node[v] (F)[above left of=E] {F};
                  \node[v] (D)[above right of=E] {D};
                  \node[v] (C)[below right of=D] {C};
                  \draw (A) -- node[e] {2} (F);
                  \draw (B) -- node[e] {2} (C);
                  \draw (B) -- node[e] {2} (E);
                  \draw (C) -- node[e] {8} (D);
                  \draw (D) -- node[e] {6} (F);
              \end{tikzpicture}
          \end{center}
\end{enumerate}

\subsection*{Zadanie 4}
(2 pkt) Napisz program, który czyta z pliku graf w następującym formacie:
(a) Pierwsza linia zawiera liczbę wierzchołków \verb|n| oraz liczbę krawędzi \verb|e|.
(b) w następnych \verb|e| liniach są po trzy liczby zadające krawędź: numer wierzchołka
startowego, numer wierzchołka docelowego, oraz długość krawędzi. Wierzchołki są numerowane liczbami od 1 do \verb|n|. \\[1em]
Na podstawie pliku tworzony jest graf w reprezentacji list sąsiedztwa. \\[1em]
Po wczytaniu grafu, program drukuje:
(a) macierz sąsiedztwa,
(b) minimalne drzewo rozpinające (algorytm Kruskala lub Prima)
(c) drzewo najkrótszych ścieżek z wierzchołka o numerze 1 (algorytm Dijkstry): dla
każdego wierzchołka (z wyjątkiem 1) drukowany jest drugi koniec krawędzi, jej długość
oraz odległość całkowita od wierzchołka 1. \\[1em]
Format wydruku drzew w punktach (b) i (c) jest taki sam jak dla grafu wejściowego. \\[1em]
Za punkt wyjścia możesz użyć program \verb|graph2.cc| zamieszczony w materiałach do wykładu 11.

\subsection*{Zadanie 5}
(2 pkt) Napisz program, który (np. metodą prób i błędów, przeszukując “szachownicę” w
głąb) znajduje drogę konika szachowego po szachownicy o podanych wymiarach, taką
że każde pole jest odwiedzone dokładnie raz. Wypróbuj program dla kwadratowych szachownic o boku od 5 do 20.

\end{document}
