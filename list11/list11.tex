\documentclass{article}

\usepackage{amsmath}
\usepackage{amssymb}
\usepackage{enumitem}
\usepackage[polish]{babel}
\usepackage[T1]{fontenc}
\usepackage[utf8]{inputenc}
\usepackage[margin=0.8in]{geometry}
\usepackage[edges]{forest}
\usepackage{listings}
\usepackage{multicol}
\usepackage{tikz}
\usetikzlibrary{arrows.meta, matrix}

\begin{document}

\title{Algorytmy i struktury danych}
\author{}
\date{}
\maketitle

\section*{Lista zadań 11}

\subsection*{Zadanie 1}
Regularne drzewo binarne to takie, które nie ma węzłów z jednym dzieckiem. Udowodnij, że drzewo
binarne, które nie jest regularne, nie może odpowiadać optymalnemu kodowi prefiksowemu.
\noindent \\[1em]
Załóżmy nie wprost, że drzewo binarne, które nie jest regularne, odpowiada optymalnemu kodowi
prefiksowemu. W takim drzewie istnieje węzeł, który ma tylko jedno dziecko. Węzeł ten może być
usunięty, a jego dziecko może zostać podłączone do jego rodzica. W ten sposób otrzymujemy drzewo
binarne, które odpowiada optymalnemu kodowi prefiksowemu, ale jest regularne. Otrzymaliśmy
sprzeczność, więc założenie musi być fałszywe. Drzewo binarne, które nie jest regularne, nie może
odpowiadać optymalnemu kodowi prefiksowemu.

\subsection*{Zadanie 2}
Czy kody Huffmana są wyznaczone jednoznacznie dla każdego tekstu? Dlaczego?
\noindent \\[1em]
Tak, dla tego samego algorytmu algorytm Huffmana wyznaczy te same kody dla tego samego tekstu.
Podczas tworzenia drzewa Huffmana, węzły są sortowane po częstości występowania znaku i następnie
drzewa o najmniejszej częstości są scalane w jedno drzewo. Jeśli istnieją dwa drzewa o tej samej
częstości, to węzły są sortowane po kolejności występowania znaku w alfabecie. W związku z tym
algorytm Huffmana zawsze wyznaczy te same kody dla tego samego tekstu.

\subsection*{Zadanie 3}
Dla podanego tekstu ”bababacacadaaasadaca”.
\begin{enumerate}[label=(\alph*)]
    \item Zasymuluj działanie algorytmu generującego kody Huffmana, narysuj otrzymane
          drzewo kodów, wypisz kody poszczególnych znaków, oraz zakodowany tekst.
    \item Oblicz, o ile bitów otrzymana reprezentacja tekstu będzie krótsza od reprezentacji
          otrzymanej za pomocą kodów o stałej długości.
    \item Mając dane drzewo kodów i zakodowany tekst wykonaj dekodowanie (zaznaczaj w
          ciągu bitów kreską gdzie kończą się kody poszczególnych znaków).
    \item Zobacz, jak deszyfrowany tekst zmieni się, gdy zmienisz pierwszy bit zaszyfrowanej
          wersji na przeciwny.
\end{enumerate}

\subsection*{Zadanie 4}
Jaki jest optymalny kod Huffmana, dla zbioru częstości opartego na początkowych n = 8
liczbach Fibonacciego: a:1, b:1, c:2, d:3, e:5, f:8, h:21?
Uogólnij odpowiedź na przypadek dowolnego n.

\subsection*{Zadanie 5}
Jaki jest optymalny kod Huffmana, dla zbioru częstości opartego na liczbach: a:11, b:12,
c:13, d:14, e:15, f:18, h:19? Porównaj długość zakodowanego tekstu z długością tekstu
zakodowanego przy pomocy kodów stałej długości.
Uogólnij odpowiedź na przypadek $n = 2^k$ liter w alfabecie, gdzie dodatkowo maksymalna
ilość wystąpień jest mniejsza od dwukrotności minimalnej ilości wystąpień.

\subsection*{Zadanie 6}
Udowodnij, że długość (ilość bitów) zaszyfrowanego tekstu, jest sumą liczb występujących w
wewnętrznych węzłach drzewa kodów (nie liściach).

\subsection*{Zadanie 7}
Narysuj na kartce sieć sortującą $n$ liczb dla $n$ = 2, 4, 8, 16. Powinna to być opisana na
wykładzie sieć implementująca równoległą wersję algorytmu mergesort, działająca w
czasie $O((log n)^2)$. Prześledź działanie sieci o $n = 8$ dla ciągu wejściowego: 8 4 2 3 7 5 6
1, rysując jakie liczby wchodzą i wychodzą z każdego komparatora.


\end{document}
