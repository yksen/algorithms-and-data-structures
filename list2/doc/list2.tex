\documentclass{article}

\usepackage{amsmath}
\usepackage{enumitem}
\usepackage[polish]{babel}
\usepackage[T1]{fontenc}
\usepackage[utf8]{inputenc}
\usepackage[margin=1.5in]{geometry}
\usepackage[edges]{forest}
\usepackage{listings}

\begin{document}

\title{Algorytmy i struktury danych}
\author{}
\date{}
\maketitle

\section*{Lista zadań 2}

\subsection*{Zadanie 1}
Ile trzeba porównań, by znaleźć element $x$ w nieuporządkowanej tablicy \lstinline{t} o rozmiarze
$n$. Oblicz wartość średnią i wariancję zakładając, że element $x$ może znajdować się z
jednakowym prawdopodobieństwem, pod dowolnym indeksem tablicy.
\begin{gather*}
    E(X) = \sum_{i=1}^{n} x_ip_i = \sum_{i=1}^{n} i \cdot \frac{1}{n} = \frac{\left(\frac{1}{n}+\frac{n}{n}\right)n}{2} = \frac{n+1}{2} \\
    Var(X) = E(X^2) - E(X)^2 = \frac{n^2+1}{2} - \left(\frac{n+1}{2}\right)^2 = \frac{2n^2+2}{4} - \frac{n^2+2n+1}{4} = \frac{n^2-2n+1}{4}
\end{gather*}

\end{document}