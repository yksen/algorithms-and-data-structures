\documentclass{article}

\usepackage{amsmath}
\usepackage{amssymb}
\usepackage{enumitem}
\usepackage[polish]{babel}
\usepackage[T1]{fontenc}
\usepackage[utf8]{inputenc}
\usepackage[margin=1.5in]{geometry}
\usepackage[edges]{forest}
\usepackage{listings}
\usepackage{multicol}

\begin{document}

\title{Algorytmy i struktury danych}
\author{}
\date{}
\maketitle

\section*{Lista zadań 4}

\subsection*{Zadanie 1}
Skorzystaj z metody rekurencji uniwersalnej i podaj dokładne asymptotyczne
oszacowania dla następujących rekurencji:
\begin{enumerate}[label=(\alph*)]
    \item $n^{1/2} = \Theta(n^{log_4 2}) \implies T(n) = 2T(n/4) + \sqrt{n} = \Theta(n^{1/2}logn)$
    \item $n = \Omega(n^{log_4 3+\epsilon}) \text{ dla } \epsilon < 0.2 \implies T(n) = 3T(n/4) + n = \Theta(n)$
    \item $n^{3/2} = \Theta(n^{log_4 8}) \implies T(n) = 8T(n/4) + n\sqrt{n} = \Theta(n^{3/2}logn)$
    \item \begin{gather*}
              T(n) = 2T\left(n^\frac{1}{2}\right) + 1 \\
              m = \log n \text{, } U(m) = T\left(e^m\right) = T\left(e^{\log n}\right) = T(n) \\
              U(m) = 2T\left(e^{\log n^\frac{1}{2}}\right) + 1 = 2T\left(e^{\frac{1}{2}\log n}\right) + 1 = 2T\left(e^{\frac{m}{2}}\right) + 1 = 2U\left(\frac{m}{2}\right) + 1 \\
              1 = O\left(m^{log_2 2-\epsilon}\right) \text{ dla } \epsilon \leq 1 \implies U(m) = \Theta(m) \\
              T(n) = U(m) = U(\log n) = \Theta(m) = \Theta(\log n)
          \end{gather*}
\end{enumerate}

\subsection*{Zadanie 2}
Czas działania algorytmu $A$ opisany jest przez rekurencję $T(n) = 7T(n/2) + n^2$.
Algorytm konkurencyjny $A'$ ma czas działania $T '(n) = aT'(n/4) + n^2$. Jaka jest największa
liczba całkowita $a$, przy której $A'$ jest asymptotycznie szybszy niż $A$?
\begin{gather*}
    n^2 = O\left(n^{log_2 7-\epsilon}\right) \text{ dla } \epsilon \leq 0.80 \implies T(n) = 7T(n/2) + n^2 = \Theta\left(n^{log_2 7}\right) \\
    n^2 = O\left(n^{log_4 a-\epsilon}\right) \implies T'(n) = aT'(n/4) + n^2 = \Theta\left(n^{log_4 a}\right) \\
    n^{log_4a} < n^{log_2 7} \\
    n^{\frac{1}{2}log_2a} < n^{log_2 7}
\end{gather*}
\begin{equation*}
    T'(n) = \begin{cases}
        \Theta\left(n^2\right)                                           & \text{ dla } a \in [1, 15]      \\
        \Theta\left(n^2logn\right)                                       & \text{ dla } a = 16             \\
        \Theta\left(n^{log_4 a}\right)                                   & \text{ dla } a \in [17, \infty) \\
        \Theta\left(n^{log_4 49}\right) = \Theta\left(n^{log_2 7}\right) & \text{ dla } a = 49
    \end{cases}
\end{equation*}

\subsection*{Zadanie 3}
Rozważmy warunek regularności $af(n/b) \leq cf(n)$ dla pewnej stałej $c \leq 1$, który jest
częścią przypadku 3 twierdzenia o rekurencji uniwersalnej. Podaj przykład prostej funkcji
$f(n)$, które spełnia wszystkie warunki twierdzenia o rekurencji uniwersalnej z wyjątkiem
warunku regularności.
\begin{gather*}
    af(n/b) \leq cf(n) \text{, } a \geq 1 \text{, } b > 1 \text{, } c \leq 1 \\
    T(n) = T(n/2) + \sin \frac{n \pi}{2} + 2 \cdot \sqrt{n} \\
    \sin\left(\frac{n \pi}{2}\right) \cdot \sqrt{n} \leq c \sin\left(n \pi\right) + 2 \cdot \sqrt{2n} \\
    (1 + 2) \cdot \sqrt{n} \leq c (0 + 2) \cdot \sqrt{2n}
\end{gather*}

\subsection*{Zadanie 4}
Zasymuluj działanie polifazowego mergesorta dla tablicy:
\begin{center}
    \verb|{9,22,6,19,21,14,10,17,3,5,60,30,29,1,8,7,6,15,12}|.
\end{center}
W sortowaniu polifazowym na każdym etapie sortowania scala się sąsiadujące podciągi rosnące, to znaczy:
w pierwszym przebiegu \verb|{9,22}| z \verb|{6,19,21}|, \verb|{14}| z \verb|{10,17}| itd..
\begin{center}
    \verb+{9,22|6,19,21|14|10,17|3,5,60|30|29|1,8|7|6,15|12}+ \\
    \verb+{6,9,19,21,22|10,14,17|3,5,30,60|1,8,29|6,7,15|12}+ \\
    \verb+{6,9,10,14,17,19,21,22|1,3,5,8,29,30,60|6,7,12,15}+ \\
    \verb+{1,3,5,6,8,9,10,14,17,19,21,22,29,30,60|6,7,12,15}+ \\
    \verb+{1,3,5,6,6,7,8,9,10,12,14,15,17,19,21,22,29,30,60}+
\end{center}

\subsection*{Zadanie 5}
\begin{enumerate}[label=(\alph*)]
    \item Czy tablica posortowana malejąco jest kopcem?
          \begin{center}
              \begin{forest}
                  for tree={circle, draw, minimum size=3ex, inner sep=1pt, s sep=7mm, anchor=south, fill=green!50}
                  [7[6[4][3]][5[2][1]]]
              \end{forest}
          \end{center}
    \item Czy ciąg \verb+{23,17,14,6,13,10,1,5,7,12}+ jest kopcem?
          \begin{center}
              \begin{forest}
                  for tree={circle, draw, minimum size=3ex, inner sep=1pt, s sep=7mm, anchor=south, fill=green!50}
                  [23[17[6[5][7, fill=red!50]][13[12][, no edge, draw=none, fill=none]]][14[10[, no edge, draw=none, fill=none][, no edge, draw=none, fill=none]][1[, no edge, draw=none, fill=none][, no edge, draw=none, fill=none]]]]
              \end{forest}
          \end{center}
\end{enumerate}

\pagebreak
\subsection*{Zadanie 6}
Zilustruj działanie procedury \verb+buildheap+ dla ciągu \verb+{5,3,17,10,84,19,6,22,9,14,3}+.
Narysuj na kartce wygląd tablicy i kopca po każdym wywołaniu procedury przesiej.
\begin{multicols*}{2}
    \begin{center}
        \verb+{5,3,17,10,84,19,6,22,9,14,3}+ \\[12ex]
        \verb+{5,3,17,22,84,19,6,10,9,14,3}+ \\
        \verb+{5,3,17,22,84,19,6,10,9,14,3}+ \\[9ex]
        \verb+{5,3,19,22,84,17,6,10,9,14,3}+ \\
        \verb+{5,3,19,22,84,17,6,10,9,14,3}+ \\[9ex]
        \verb+{5,84,19,22,3,17,6,10,9,14,3}+ \\
        \verb+{5,84,19,22,14,17,6,10,9,3,3}+ \\
        \verb+{5,84,19,22,14,17,6,10,9,3,3}+ \\[16ex]
        \verb+{84,5,19,22,14,17,6,10,9,3,3}+ \\
        \verb+{84,22,19,5,14,17,6,10,9,3,3}+ \\
        \verb+{84,22,19,10,14,17,6,5,9,3,3}+ \\
        \verb+{84,22,19,10,14,17,6,5,9,3,3}+
    \end{center}
    \columnbreak
    \begin{center}
        \begin{forest}
            for tree={circle, draw, minimum size=3ex, inner sep=1pt, s sep=7mm, l sep=0mm, l=0mm, anchor=south}
            [5[3[10[22][9]][84, fill=green!50[14][3]]][17[19][6]]]
        \end{forest}
        \hrule
        \begin{forest}
            for tree={circle, draw, minimum size=3ex, inner sep=1pt, s sep=7mm, l sep=0mm, l=0mm, anchor=south}
            [5[3[22, fill=green!50[10, fill=green!50][9]][84[14][3]]][17[19][6]]]
        \end{forest}
        \hrule
        \begin{forest}
            for tree={circle, draw, minimum size=3ex, inner sep=1pt, s sep=7mm, l sep=0mm, l=0mm, anchor=south}
            [5[3[22[10][9]][84[14][3]]][19, fill=green!50[17, fill=green!50][6]]]
        \end{forest}
        \hrule
        \begin{forest}
            for tree={circle, draw, minimum size=3ex, inner sep=1pt, s sep=7mm, l sep=0mm, l=0mm, anchor=south}
            [5[84, fill=green!50[22[10][9]][3, fill=green!50[14][3]]][19[17][6]]]
        \end{forest}
        \begin{forest}
            for tree={circle, draw, minimum size=3ex, inner sep=1pt, s sep=7mm, l sep=0mm, l=0mm, anchor=south}
            [5[84[22[10][9]][14, fill=green!50[3, fill=green!50][3]]][19[17][6]]]
        \end{forest}
        \hrule
        \begin{forest}
            for tree={circle, draw, minimum size=3ex, inner sep=1pt, s sep=7mm, l sep=0mm, l=0mm, anchor=south}
            [84, fill=green!50[5, fill=green!50[22[10][9]][14[3][3]]][19[17][6]]]
        \end{forest}
        \begin{forest}
            for tree={circle, draw, minimum size=3ex, inner sep=1pt, s sep=7mm, l sep=0mm, l=0mm, anchor=south}
            [84[22, fill=green!50[5, fill=green!50[10][9]][14[3][3]]][19[17][6]]]
        \end{forest}
        \begin{forest}
            for tree={circle, draw, minimum size=3ex, inner sep=1pt, s sep=7mm, l sep=0mm, l=0mm, anchor=south}
            [84[22[10, fill=green!50[5, fill=green!50][9]][14[3][3]]][19[17][6]]]
        \end{forest}
    \end{center}
\end{multicols*}

\subsection*{Zadanie 8}
Udowodnij, że wysokość kopca $n$-elementowego wynosi $\lfloor\log_2 n\rfloor + 1$.
\begin{gather*}
    \text{Maksymalna ilość węzłów w kopcu o wysokości $h$:} \\
    n(h) = 2^h - 1 \\
    n(h-1) = 2^{h - 1} - 1 \\
    \text{Minimalna ilość węzłów w kopcu o wysokości $h$:} \\
    n(h-1) + 1 = 2^{h - 1} - 1 + 1 = 2^{h - 1} \\
    \text{Ilość węzłów w kopcu o wysokości $h$:} \\
    2^{h - 1} \leq n < 2^h \\
    h-1 \leq \log_2 n < h \\
    h \leq \log_2 n + 1 < h + 1 \\
    \lfloor\log_2 n\rfloor + 1
\end{gather*}

\subsection*{Zadanie 11}
Niech $F_n$ oznacza ilość różnych kształtów drzew binarnych o $n$ węzłach. Rysując drzewa,
łatwo sprawdzić, że $F_0 = 1, F_1 = 1, F_2 = 2, F_3 = 5$, itd. Nie korzystając z internetu:
\begin{enumerate}[label=(\alph*)]
    \item Znajdź wzór wyrażający $F_n$ przez $F_0, F_1, F_2, \dots, F_{n-1}$ dla $n = 2, 3, 4$ a potem ogólnie.
          \begin{gather*}
              ?
          \end{gather*}
    \item Zaprojektuj (na kartce) procedurę, która oblicza kolejne wyrazy ciągu $F_n$, zapisuje
          je w tablicy i korzysta z nich przy obliczaniu następnych wyrazów.
    \item Przeanalizuj ile mnożeń trzeba wykonać, by obliczyć wyrazy od $F_1$ do $F_n$. Czy da
          się ją zapisać w postaci $O(n^k)$ dla pewnego $k$?
    \item Jaka byłaby złożoność algorytmu rekurencyjnego, który nie korzysta z wartości
          zapisanych w tablicy, tylko oblicza je ponownie. Czy da się ją zapisać jako $O(n^k)$?
\end{enumerate}

\end{document}