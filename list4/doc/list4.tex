\documentclass{article}

\usepackage{amsmath}
\usepackage{enumitem}
\usepackage[polish]{babel}
\usepackage[T1]{fontenc}
\usepackage[utf8]{inputenc}
\usepackage[margin=1.5in]{geometry}
\usepackage[edges]{forest}
\usepackage{listings}

\begin{document}

\title{Algorytmy i struktury danych}
\author{}
\date{}
\maketitle

\section*{Lista zadań 4}

\subsection*{Zadanie 1}
Skorzystaj z metody rekurencji uniwersalnej i podaj dokładne asymptotyczne
oszacowania dla następujących rekurencji:
\begin{enumerate}[label=(\alph*)]
    \item $n^{1/2} = \Theta(n^{log_4 2}) \implies T(n) = 2T(n/4) + \sqrt{n} = \Theta(n^{1/2}logn)$
    \item $n = \Omega(n^{log_4 3+\epsilon}) \text{ dla } \epsilon < 0.2 \implies T(n) = 3T(n/4) + n = \Theta(n)$
    \item $n^{3/2} = \Theta(n^{log_4 8}) \implies T(n) = 8T(n/4) + n\sqrt{n} = \Theta(n^{3/2}logn)$
    \item \begin{gather*}
              T(n) = 2T\left(n^\frac{1}{2}\right) + 1 \\
              m = \log n \text{, } U(m) = T\left(e^m\right) = T\left(e^{\log n}\right) = T(n) \\
              U(m) = 2T\left(e^{\log n^\frac{1}{2}}\right) + 1 = 2T\left(e^{\frac{1}{2}\log n}\right) + 1 = 2U\left(\frac{m}{2}\right) + 1 \\
              1 = O\left(m^{log_2 2-\epsilon}\right) \text{ dla } \epsilon \leq 1 \implies U(m) = \Theta(m) \\
              T(n) = U(m) = U(\log n) = \Theta(m) = \Theta(\log n)
          \end{gather*}
\end{enumerate}

\subsection*{Zadanie 2}
Czas działania algorytmu $A$ opisany jest przez rekurencję $T(n) = 7T(n/2) + n^2$.
Algorytm konkurencyjny $A'$ ma czas działania $T '(n) = aT'(n/4) + n^2$. Jaka jest największa
liczba całkowita $a$, przy której $A'$ jest asymptotycznie szybszy niż $A$?
\begin{gather*}
    n^2 = O\left(n^{log_2 7-\epsilon}\right) \text{ dla } \epsilon \leq 0.80 \implies T(n) = 7T(n/2) + n^2 = \Theta\left(n^{log_2 7}\right) \\
    n^2 = O\left(n^{log_4 a-\epsilon}\right) \implies T'(n) = aT'(n/4) + n^2 = \Theta\left(n^{log_4 a}\right) \\
    n^{log_4a} < n^{log_2 7} \\
    n^{\frac{1}{2}log_2a} < n^{log_2 7}
\end{gather*}
\begin{equation*}
    T'(n) = \begin{cases}
        \Theta\left(n^2\right)                                           & \text{ dla } a \in [1, 15]      \\
        \Theta\left(n^2logn\right)                                       & \text{ dla } a = 16             \\
        \Theta\left(n^{log_4 a}\right)                                   & \text{ dla } a \in [17, \infty) \\
        \Theta\left(n^{log_4 49}\right) = \Theta\left(n^{log_2 7}\right) & \text{ dla } a = 49
    \end{cases}
\end{equation*}

\subsection*{Zadanie 4}
Zasymuluj działanie polifazowego mergesorta dla tablicy:
\begin{center}
    \verb|{9,22,6,19,21,14,10,17,3,5,60,30,29,1,8,7,6,15,12}|.
\end{center}
W sortowaniu polifazowym na każdym etapie sortowania scala się sąsiadujące podciągi rosnące, to znaczy:
w pierwszym przebiegu \verb|{9,22}| z \verb|{6,19,21}|, \verb|{14}| z \verb|{10,17}| itd..
\begin{center}
    \verb+{9,22|6,19,21|14|10,17|3,5,60|30|29|1,8|7|6,15|12}+ \\
    \verb+{6,9,19,21,22|10,14,17|3,5,30,60|1,8,29|6,7,15|12}+ \\
    \verb+{6,9,10,14,17,19,21,22|1,3,5,8,29,30,60|6,7,12,15}+ \\
    \verb+{1,3,5,6,8,9,10,14,17,19,21,22,29,30,60|6,7,12,15}+ \\
    \verb+{1,3,5,6,6,7,8,9,10,12,14,15,17,19,21,22,29,30,60}+
\end{center}

\end{document}