\documentclass{article}

\usepackage{amsmath}
\usepackage{enumitem}
\usepackage[polish]{babel}
\usepackage[T1]{fontenc}
\usepackage[utf8]{inputenc}
\usepackage[margin=1.5in]{geometry}
\usepackage[edges]{forest}
\usepackage{listings}

\begin{document}

\title{Algorytmy i struktury danych}
\author{}
\date{}
\maketitle

\section*{Lista zadań 4}

\subsection*{Zadanie 1}
Skorzystaj z metody rekurencji uniwersalnej i podaj dokładne asymptotyczne
oszacowania dla następujących rekurencji:
\begin{enumerate}[label=(\alph*)]
    \item $T(n) = 2T(n/4) + \sqrt{n} = \Theta(n^{1/2}logn)$, $n^{1/2} = \Theta(n^{log_4 2})$
    \item $T(n) = 3T(n/4) + n = \Theta(n)$, $n = \Omega(n^{log_4 3})$
    \item $T(n) = 8T(n/4) + n\sqrt{n} = \Theta(n^{3/2}logn)$, $n^{3/2} = \Theta(n^{log_4 8})$
    \item \begin{gather*}
              T(n) = 2T\left(n^\frac{1}{2}\right) + 1 \\
              m = \log n \text{, } U(m) = T\left(e^m\right) = T\left(e^{\log n}\right) = T(n) \\
              U(m) = 2T\left(e^{\log n^\frac{1}{2}}\right) + 1 = 2T\left(e^{\frac{1}{2}\log n}\right) + 1 = 2U\left(\frac{m}{2}\right) + 1 \\
              U(m) = \Theta(m) \text{, } 1 = O\left(m^{log_2 2-\epsilon}\right) \text{ dla } \epsilon \leq 1 \\
              T(n) = U(m) = U(\log n) = \Theta(m) = \Theta(\log n)
          \end{gather*}
\end{enumerate}

\subsection*{Zadanie 2}
Czas działania algorytmu $A$ opisany jest przez rekurencję $T(n) = 7T(n/2) + n^2$.
Algorytm konkurencyjny $A'$ ma czas działania $T '(n) = aT'(n/4) + n^2$. Jaka jest największa
liczba całkowita $a$, przy której $A'$ jest asymptotycznie szybszy niż $A$?



\end{document}