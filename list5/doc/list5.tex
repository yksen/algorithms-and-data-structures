\documentclass{article}

\usepackage{amsmath}
\usepackage{amssymb}
\usepackage{enumitem}
\usepackage[polish]{babel}
\usepackage[T1]{fontenc}
\usepackage[utf8]{inputenc}
\usepackage[margin=1.5in]{geometry}
\usepackage[edges]{forest}
\usepackage{listings}
\usepackage{multicol}

\begin{document}

\title{Algorytmy i struktury danych}
\author{}
\date{}
\maketitle

\section*{Lista zadań 5}

\subsection*{Zadanie 2}
Ile porównań (zapisz wyniki w notacji O) wykona algorytm \verb+quicksort+ z
procedurą \verb+partition+ w wersji Hoare'a, a ile w wersji z procedurę \verb+partition+ w wersji Lomuto dla
danych: (a) posortowanych rosnąco, (b) posortowanych malejąco, (c) o identycznych kluczach?

\subsection*{Zadanie 6}
Posortuj metodą sortowania pozycyjnego liczby: 101, 345, 103, 333, 432, 132, 543, 651,
791, 532, 987, 910, 643, 641, 12, 342, 498, 987, 965, 322, 121, 431, 350. W pisemnym
rozwiązaniu pokaż, jak wygląda zawartość kolejek, za każdym razem, gdy tablica
wyjściowa jest pusta i wszystkie liczby znajdują się w kolejkach, oraz jak wygląda
tablica wyjściowa, za każdym razem, gdy sortowanie ze względu na kolejną cyfrę jest już zakończone.

\begin{align*}
    0: & \left(910, 350\right) \\
    1: & \left(101, 651, 791, 641, 121, 431\right) \\
    2: & \left(432, 132, 532, 12, 342, 322\right) \\
    3: & \left(103, 333, 543, 643\right) \\
    4: & \left(\right) \\
    5: & \left(345, 965\right) \\
    6: & \left(\right) \\
    7: & \left(987, 987\right) \\
    8: & \left(498\right) \\
    9: & \left(\right)
\end{align*}
(910, 350, 101, 651, 791, 641, 121, 431, 432, 132, 532, 12, 342, 322, 103, 333, 543, 643, 345, 965, 987, 987, 498)

\begin{align*}
    0: & \left(101, 103\right) \\
    1: & \left(910\right) \\
    2: & \left(121, 12, 322\right) \\
    3: & \left(431, 432, 132, 532, 333\right) \\
    4: & \left(641, 342, 543, 643, 345\right) \\
    5: & \left(350, 651\right) \\
    6: & \left(965\right) \\
    7: & \left(\right) \\
    8: & \left(987, 987\right) \\
    9: & \left(791, 498\right)
\end{align*}
(101, 103, 910, 121, 12, 322, 431, 432, 132, 532, 333, 641, 342, 543, 643, 345, 350, 651, 965, 987, 987, 791, 498)

\begin{align*}
    0: & \left(12\right) \\
    1: & \left(101, 103, 121, 132\right) \\
    2: & \left(\right) \\
    3: & \left(322, 333, 342, 345, 350\right) \\
    4: & \left(431, 432, 498\right) \\
    5: & \left(532, 543\right) \\
    6: & \left(641, 643, 651\right) \\
    7: & \left(791\right) \\
    8: & \left(\right) \\
    9: & \left(910, 965, 987, 987\right)
\end{align*}
(12, 101, 103, 121, 132, 322, 333, 342, 345, 350, 431, 432, 498, 532, 543, 641, 643, 651, 791, 910, 965, 987, 987)

\end{document}