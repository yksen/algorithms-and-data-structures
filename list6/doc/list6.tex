\documentclass{article}

\usepackage{amsmath}
\usepackage{amssymb}
\usepackage{enumitem}
\usepackage[polish]{babel}
\usepackage[T1]{fontenc}
\usepackage[utf8]{inputenc}
\usepackage[margin=0.8in]{geometry}
\usepackage[edges]{forest}
\usepackage{listings}
\usepackage{multicol}

\begin{document}

\title{Algorytmy i struktury danych}
\author{}
\date{}
\maketitle

\section*{Lista zadań 5}

\subsection*{Zadanie 1}
Jakie informacje przechowujemy w węźle drzewa czerwono-czarnego?
Zadeklaruj strukturę \verb+RBTnode+ tak, by dziedziczyła z \verb+BSTnode+. Podaj definicję drzewa czerwono czarnego.

\subsection*{Zadanie 2}
\begin{enumerate}[label=(\alph*)]
    \item Jaka może być minimalna, a jaka maksymalna ilość kluczy w drzewie czerwono-czarnym
          o ustalonej czarnej wysokości równej $h_B$?
    \item Znajdź maksymalną i minimalną wartość stosunku ilości węzłów czerwonych do
          czarnych w drzewie czerwono-czarnym.
\end{enumerate}

\subsection*{Zadanie 3}
Uzasadnij posługując się rysunkiem i opisem, że operacje wykonywane w trakcie wstawiania
do drzewa czerwono-czarnego (rotacja i przekolorowanie) nie zmieniają ilości
czarnych węzłów, na żadnej ścieżce od korzenia do liścia.

\subsection*{Zadanie 4}
\begin{enumerate}[label=(\alph*)]
    \item Narysuj poprawne drzewo czerwono-czarne w którym na lewo od korzenia jest $1$
          węzeł a na prawo $7$ węzłów.
    \item Czy istnieje poprawne drzewo czerwono-czarne, w którym na lewo od korzenia będzie
          $100$ razy mniej węzłów niż na prawo od korzenia?
\end{enumerate}

\subsection*{Zadanie 5}
W poniższym drzewie czerwono-czarnym:
\begin{center}
    \begin{forest}
        for tree={circle, minimum size=3ex, inner sep=1pt, s sep=7mm, anchor=south, fill=black, text=white},
        [5[3, fill=red[1][4]][11[9, fill=red][, no edge, fill=none]]]
    \end{forest}
\end{center}
\begin{enumerate}[label=-]
    \item wstaw do niego \verb+10+.
    \item usuń z wyjściowego drzewa \verb+1+.
\end{enumerate}

\subsection*{Zadanie 6}
(3 pkt.) Do pustego drzewa czerwono-czarnego wstaw kolejno $20$ przypadkowych kluczy.
Następnie usuń je w tej samej kolejności w jakiej wstawiałeś. Przypadkowymi kluczami
są kolejne litery Twojego nazwiska, imienia i adresu. Zadanie wykonujemy na kartce
(lub w pliku) i oddajemy prowadzącemu. Zadanie jest obowiązkowe.

\subsection*{Zadanie 7}
Analizując kod programu \verb+RBT.h+ udowodnij, że w trakcie wstawiania do drzewa czerwono-czarnego
wykonają się co najwyżej dwie rotacje. Czy tak samo jest w przypadku usuwania?

\subsection*{Zadanie 8}
Uzasadnij, że rozmiar stosu $(n = 100)$ przyjęty w procedurach \verb+insert+ i \verb+remove+ w pliku
\verb+RBnpnr.h+ nigdy nie okaże się za mały.

\end{document}