\documentclass{article}

\usepackage{amsmath}
\usepackage{amssymb}
\usepackage{enumitem}
\usepackage[polish]{babel}
\usepackage[T1]{fontenc}
\usepackage[utf8]{inputenc}
\usepackage[margin=0.8in]{geometry}
\usepackage[edges]{forest}
\usepackage{listings}[language=cpp]
\usepackage{multicol}
\usepackage{tikz}
\usetikzlibrary{shapes}
\tikzstyle{bplus}=[rectangle split, rectangle split horizontal, rectangle split ignore empty parts, draw]
\tikzstyle{every node}=[bplus]
\tikzstyle{level 1}=[sibling distance=60mm]
\tikzstyle{level 2}=[sibling distance=20mm]

\begin{document}

\title{Algorytmy i struktury danych}
\author{}
\date{}
\maketitle

\section*{Lista zadań 7}

\subsection*{Zadanie 1}
Jakie informacje przechowujemy w węźle B-drzewa? Podaj definicję B-drzewa.
\begin{lstlisting}
    struct BTree
    {
        uint32_t t;
        bool isLeaf;
        size_t n;
        int32_t *keys;
        size_t *offsets;
    };
\end{lstlisting}
\begin{enumerate}
    \item Każdy węzeł posiada $n$ kluczy, przechowywanych w kolejności niemalejącej, a także informację o tym czy jest
          on liściem.
    \item Dodatkowo każdy węzeł posiada $n+1$ wskaźników do swoich dzieci.
    \item Klucze węzła dzielą zbiór kluczy przechowywanych w jego dzieciach na $n+1$ przedziałów.
    \item Wszystkie liście znajdują się na tym samym poziomie równym wysokości drzewa $h$.
    \item Każdy węzeł, z wyjątkiem korzenia, posiada co najmniej $t-1$ kluczy.
    \item Każdy węzeł może posiadać maksymalnie $2t-1$ kluczy.
\end{enumerate}

\subsection*{Zadanie 2}
(2 pkt.) Udowodnij, że żadna z poniższych operacji wykonana na drzewie spełniającym wszystkie warunki B-drzewa, nie
prowadzi do ich naruszenia.
\begin{enumerate}[label=(\alph*)]
    \item \verb|split_child|, przesuwająca środkowy klucz (medianę) z węzła o $2t-1$ kluczach do rodzica, który ma
          mniej niż $2t-1$ kluczy, a klucze i dzieci na prawo od mediany -- do nowego brata dodanego po prawej stronie
          dzielonego węzła.
    \item \verb|unsplit_child| odwrotna do \verb|split_child|, sklejająca dwa sąsiednie węzły o minimalnej liczbie
          kluczy $t-1$ oraz klucz stojący w rodzicu między nimi w jeden nowy węzeł. Zakładamy, że rodzic ma co
          najmniej $t$ kluczy lub jest korzeniem.
    \item \verb|borrow_from_sibling|, rotacja przenosząca do węzła o minimalnej $t-1$ liczbie kluczy, który ma
          prawego brata z co najmniej $t$ kluczami, klucz stojący w rodzicu między braćmi i wpisująca na jego miejsce
          jego miejsce pierwszy klucz brata. Jakie operacje na dzieciach należy dodatkowo wykonać?
\end{enumerate}

\subsection*{Zadanie 3}
W B-drzewie o $t=10$ podaj wzory i wyniki numeryczne określające:
\begin{enumerate}[label=(\alph*)]
    \item ile kluczy może zawierać korzeń (podaj przedział),
          \begin{equation*}
              n_{minr} = 1 \qquad n_{max} = 2t-1 = 19 \qquad \langle 1, 19 \rangle
          \end{equation*}
    \item ile dzieci może mieć korzeń (podaj przedział),
          \begin{equation*}
              n_{minr} = 2 \qquad n_{max} = 2t-1+1 = 20 \qquad \{0\} \cup \langle 2, 20 \rangle
          \end{equation*}
    \item ile kluczy może mieć potomek korzenia (podaj przedział),
          \begin{equation*}
              n_{min} = t-1 = 9 \qquad n_{max} = 2t-1 = 19 \qquad \langle 9, 19 \rangle
          \end{equation*}
    \item ile dzieci może mieć potomek korzenia (podaj przedział),
          \begin{equation*}
              n_{min} = t = 10 \qquad n_{max} = 2t-1+1 = 20 \qquad \langle 10, 20 \rangle
          \end{equation*}
    \item ile maksymalnie węzłów może być na $k$-tym poziomie (przyjmując, że korzeń to poziom $0$)
          \begin{equation*}
              n_{max}(k) = (2t)^k \qquad \text{dla } t=10 \implies n_{max}(k) = 20^k
          \end{equation*}
    \item ile łącznie kluczy może być na $k$-tym poziomie (podaj przedział).
          \begin{align*}
              n_{min}(k) = & (2t)^{k-1}(t-1) & \langle (2t)^{k-1}(t-1), & (2t)^k(2t-1) \rangle \\
              n_{max}(k) = & (2t)^k(2t-1)    & \langle 20^{k-1}9,       & 20^k19 \rangle
          \end{align*}
\end{enumerate}

\subsection*{Zadanie 4}
Jaka jest minimalna, a jaka maksymalna liczba kluczy w B-drzewie mającym $h$ poziomów, przy ustalonej wartości
parametru $t$ (patrz Cormen).

\subsection*{Zadanie 5}
Podano na rysunku B-drzewo o $t=2$:
\begin{center}
    \begin{tikzpicture}
        \node {9} [->]
        child {node {7}
                child {node {6}}
                child {node {8}}
            }
        child {node {11 \nodepart{two} 14 \nodepart{three} 19}
                child {node {10}}
                child {node {12 \nodepart{two} 13}}
                child[sibling distance=25mm] {node {15 \nodepart{two} 16 \nodepart{three} 17}}
                child[sibling distance=20mm] {node {20}}
            };
    \end{tikzpicture}
    \begin{enumerate}[label=-]
        \item usuń z tego drzewa 7.
    \end{enumerate}
    \begin{tikzpicture}
        \node {9} [->]
        child {node {8}
                child {node {6}}
                child {node {x}}
            }
        child {node {11 \nodepart{two} 14 \nodepart{three} 19}
                child {node {10}}
                child {node {12 \nodepart{two} 13}}
                child[sibling distance=25mm] {node {15 \nodepart{two} 16 \nodepart{three} 17}}
                child[sibling distance=20mm] {node {20}}
            };
    \end{tikzpicture}\\[1em]
    \begin{tikzpicture}
        \node {9} [->]
        child {node {x}
                child {node {6 \nodepart{two} 8}}
            }
        child {node {11 \nodepart{two} 14 \nodepart{three} 19}
                child {node {10}}
                child {node {12 \nodepart{two} 13}}
                child[sibling distance=25mm] {node {15 \nodepart{two} 16 \nodepart{three} 17}}
                child[sibling distance=20mm] {node {20}}
            };
    \end{tikzpicture}\\[1em]
    \begin{tikzpicture}
        \node {11} [->]
        child {node {9}
                child {node {6 \nodepart{two} 8}}
                child {node {10}}
            }
        child {node {14 \nodepart{two} 19}
                child {node {12 \nodepart{two} 13}}
                child[sibling distance=25mm] {node {15 \nodepart{two} 16 \nodepart{three} 17}}
                child[sibling distance=20mm] {node {20}}
            };
    \end{tikzpicture}
    \begin{enumerate}[label=-]
        \item do drzewa widocznego powyżej dodaj 18.
    \end{enumerate}
    \begin{tikzpicture}
        \node {9} [->]
        child {node {7}
                child {node {6}}
                child {node {8}}
            }
        child {node {11 \nodepart{two} 14}
                child {node {10}}
                child {node {12 \nodepart{two} 13}}
            }
        child {node {19}
                child[sibling distance=25mm] {node {15 \nodepart{two} 16 \nodepart{three} 17}}
                child[sibling distance=20mm] {node {20}}
            };
    \end{tikzpicture}
\end{center}

\subsection*{Zadanie 6}
(2 pkt.) Do pustego B-drzewa o $t=2$ wstaw kolejno $22$ litery swojego imienia i nazwiska oraz adresu. Następnie usuń w
tej samej kolejności w jakiej były wstawiane.

\subsection*{Zadanie 7}
Narysuj B-drzewo o $t=3$ zawierające dokładnie $17$ kluczy na trzech poziomach: korzeń, jego dzieci i wnuki. Następnie
usuń z tego drzewa korzeń.

\end{document}