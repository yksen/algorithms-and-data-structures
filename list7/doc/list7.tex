\documentclass{article}

\usepackage{amsmath}
\usepackage{amssymb}
\usepackage{enumitem}
\usepackage[polish]{babel}
\usepackage[T1]{fontenc}
\usepackage[utf8]{inputenc}
\usepackage[margin=0.8in]{geometry}
\usepackage[edges]{forest}
\usepackage{listings}[language=cpp]
\usepackage{multicol}
\usepackage{tikz}
\usetikzlibrary{shapes}

\begin{document}

\title{Algorytmy i struktury danych}
\author{}
\date{}
\maketitle

\section*{Lista zadań 7}

\subsection*{Zadanie 1}
Jakie informacje przechowujemy w węźle B-drzewa? Podaj definicję B-drzewa.
\begin{lstlisting}
    struct BTree
    {
        uint32_t t;
        bool isLeaf;
        size_t n;
        int32_t *keys;
        BTree **children;
    };
\end{lstlisting}
\begin{enumerate}
    \item Każdy węzeł posiada $n$ kluczy, przechowywanych w kolejności niemalejącej, a także informację o tym czy jest
          on liściem.
    \item Dodatkowo każdy węzeł posiada $n+1$ wskaźników do swoich dzieci.
    \item Klucze węzła dzielą zbiór kluczy przechowywanych w jego dzieciach na $n+1$ przedziałów.
    \item Wszystkie liście znajdują się na tym samym poziomie równym wysokości drzewa $h$.
    \item Każdy węzeł, z wyjątkiem korzenia, posiada co najmniej $t-1$ kluczy.
    \item Każdy węzeł może posiadać maksymalnie $2t-1$ kluczy.
\end{enumerate}

\subsection*{Zadanie 2}
(2 pkt.) Udowodnij, że żadna z poniższych operacji wykonana na drzewie spełniającym wszystkie warunki B-drzewa, nie
prowadzi do ich naruszenia.
\begin{enumerate}[label=(\alph*)]
    \item \verb|split_child|, przesuwająca środkowy klucz (medianę) z węzła o $2t-1$ kluczach do rodzica, który ma
          mniej niż $2t-1$ kluczy, a klucze i dzieci na prawo od mediany -- do nowego brata dodanego po prawej stronie
          dzielonego węzła.
    \item \verb|unsplit_child| odwrotna do \verb|split_child|, sklejająca dwa sąsiednie węzły o minimalnej liczbie
          kluczy $t-1$ oraz klucz stojący w rodzicu między nimi w jeden nowy węzeł. Zakładamy, że rodzic ma co
          najmniej $t$ kluczy lub jest korzeniem.
    \item \verb|borrow_from_sibling|, rotacja przenosząca do węzła o minimalnej $t-1$ liczbie kluczy, który ma
          prawego brata z co najmniej $t$ kluczami, klucz stojący w rodzicu między braćmi i wpisująca na jego miejsce
          jego miejsce pierwszy klucz brata. Jakie operacje na dzieciach należy dodatkowo wykonać?
\end{enumerate}

\subsection*{Zadanie 3}
W B-drzewie o $t=10$ podaj wzory i wyniki numeryczne określające:
\begin{enumerate}[label=(\alph*)]
    \item ile kluczy może zawierać korzeń (podaj przedział),
    \item ile dzieci może mieć korzeń (podaj przedział),
    \item ile kluczy może mieć potomek korzenia (podaj przedział),
    \item ile dzieci może mieć potomek korzenia (podaj przedział),
    \item ile maksymalnie węzłów może być na $k$-tym poziomie (przyjmując, że korzeń to poziom $0$)
    \item ile łącznie kluczy może być na $k$-tym poziomie (podaj przedział).
\end{enumerate}

\subsection*{Zadanie 4}
Jaka jest minimalna, a jaka maksymalna liczba kluczy w B-drzewie mającym $h$ poziomów, przy ustalonej wartości
parametru $t$ (patrz Cormen).

\subsection*{Zadanie 5}
Podano na rysunku B-drzewo o $t=2$:
\begin{center}
    \begin{tikzpicture}
        \tikzstyle{bplus}=[rectangle split, rectangle split horizontal, rectangle split ignore empty parts, draw]
        \tikzstyle{every node}=[bplus]
        \tikzstyle{level 1}=[sibling distance=60mm]
        \tikzstyle{level 2}=[sibling distance=15mm]
        \node {9} [->]
        child {node {7}
                child {node {6}}
                child {node {8}}
            }
        child {node {11 \nodepart{two} 14 \nodepart{three} 19}
                child {node {10}}
                child {node {12 \nodepart{two} 13}}
                child[sibling distance=25mm] {node {15 \nodepart{two} 16 \nodepart{three} 17}}
                child[sibling distance=20mm] {node {20}}
            };
    \end{tikzpicture}
    \begin{enumerate}[label=-]
        \item usuń z tego drzewa 7.
        \item do drzewa widocznego powyżej dodaj 18.
    \end{enumerate}
\end{center}

\subsection*{Zadanie 6}
(2 pkt.) Do pustego B-drzewa o $t=2$ wstaw kolejno $22$ litery swojego imienia i nazwiska oraz adresu. Następnie usuń w
tej samej kolejności w jakiej były wstawiane.

\subsection*{Zadanie 7}
Narysuj B-drzewo o $t=3$ zawierające dokładnie $17$ kluczy na trzech poziomach: korzeń, jego dzieci i wnuki. Następnie
usuń z tego drzewa korzeń.

\end{document}