\documentclass{article}

\usepackage{amsmath}
\usepackage{amssymb}
\usepackage{enumitem}
\usepackage[polish]{babel}
\usepackage[T1]{fontenc}
\usepackage[utf8]{inputenc}
\usepackage[margin=0.8in]{geometry}
\usepackage[edges]{forest}
\usepackage{listings}[language=cpp]
\usepackage{multicol}
\usepackage{tikz}

\begin{document}

\title{Algorytmy i struktury danych}
\author{}
\date{}
\maketitle

\section*{Lista zadań 8}

\subsection*{Zadanie 1}
\begin{enumerate}[label=(\alph*)]
	\item Czym różni się haszowanie łańcuchowe od otwartego? \\[1ex]
	      W przypadku haszowania łańcuchowego elementy przechowywane są na liście związanej z danym indeksem tablicy.
	      Natomiast w haszowaniu otwartym elementy przechowywane są bezpośrednio w tablicy, a w przypadku kolizji
	      są umieszczane w innych komórkach tablicy.
	\item Czym różnią się dwie wersje haszowania otwartego: haszowanie liniowe i haszowanie podwójne? \\[1ex]
	      W przypadku haszowania liniowego, gdy wystąpi kolizja, element jest umieszczany w następnej wolnej komórce
	      tablicy. Natomiast w przypadku haszowania podwójnego, gdy wystąpi kolizja, element jest umieszczany w komórce
	      określonej przez drugą funkcję haszującą.
	\item Dla tablicy z haszowaniem podwójnym o rozmiarze $m=11$ i funkcjach haszujących:
	      $h_1(x)=x \mathop{\mathrm{mod}} 11$  oraz $h_2(x)= x \mathop{\mathrm{mod}} 10+1$ wyznacz ciąg kontrolny
	      dla liczby 23. Jak wyglądałby ten ciąg w przypadku haszowania liniowego?
	      \begin{center}
		      \begin{tabular}{c | c}
			      \textbf{haszowanie podwójne}     & \textbf{haszowanie liniowe}      \\
			      \hline
			      1, 5, 9, 2, 6, 10, 3, 7, 0, 4, 8 & 1, 2, 3, 4, 5, 6, 7, 8, 9, 10, 0
		      \end{tabular}
	      \end{center}
\end{enumerate}

\subsection*{Zadanie 2}
(2pkt) Porównaj jaka będzie łączna liczba kolizji, gdy do tablicy z poprzedniego zadania wstawimy kolejno liczby:
22, 66, 44, 23, 35,  używając: (a) haszowania liniowego, (b) haszowania dwukrotnego, (c) haszowania łańcuchowego. \\[1ex]
\noindent
Następnie w każdym z wariantów sprawdź, jaka będzie łączna liczba porównań kluczy, gdy w gotowej tablicy wywołamy
kolejno procedurę FIND (a) dla każdego elementu obecnego w tablicy, (b) dla elementów: 24 i 34, których nie ma w tablicy.
\begin{center}
	\begin{tabular}{c|c|c|c}
		\textbf{Metoda}       & \textbf{Liczba kolizji} & \textbf{Liczba porównań (a)} & \textbf{Liczba porównań (b)} \\
		\hline
		haszowanie liniowe    & 7                       & 12                           & 6                            \\
		haszowanie podwójne   & 2                       & 7                            & 4                            \\
		haszowanie łańcuchowe & 2                       & 8                            & 2
	\end{tabular}
\end{center}

\subsection*{Zadanie 3}
Narysuj przykładowe kopce dwumianowe o 12, 14, 15 i 16 węzłach. Na rysunku uwzględnij wartości kluczy oraz stopnie
węzłów. W kopcu 12-elementowym zaznacz dodatkowo strzałki (najlepiej w różnych kolorach) przedstawiające wskaźniki
na ojca, syna i brata.

\subsection*{Zadanie 4}
Zilustruj działanie operacji UNION łączącej kopiec dwumianowy o 14 węzłach z kopcem
dwumianowym o 11 węzłach. Przyjmij dowolne wartości kluczy spełniające warunek kopca.

\subsection*{Zadanie 5}
Udowodnij, że:
\begin{enumerate}[label=(\alph*)]
	\item drzewo dwumianowe rzędu $n$ ma $2^n$ węzłów.
	\item na $k$-tym poziomie drzewa dwumianowego rzędu $n$ znajduje się dokładnie $n \choose k$ węzłów.
\end{enumerate}

\subsection*{Zadanie 6}
Napisz funkcję \verb`int ile_drzew_w_kopcu(int n)` wyliczającą ile jest drzew dwumianowych w kopcu
dwumianowym zawierającym $n$ kluczy.

\subsection*{Zadanie 7}
\begin{enumerate}[label=(\alph*)]
	\item Do pustego kopca dwumianowego wstaw (INSERT) kolejno: 1, 12, 3, 14, 5, 16, 7, 20, 25 13, 8
	\item Dla otrzymanego kopca dwukrotnie wykonaj operację GETMAX.
\end{enumerate}

\end{document}
