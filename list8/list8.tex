\documentclass{article}

\usepackage{amsmath}
\usepackage{amssymb}
\usepackage{enumitem}
\usepackage[polish]{babel}
\usepackage[T1]{fontenc}
\usepackage[utf8]{inputenc}
\usepackage[margin=0.8in]{geometry}
\usepackage[edges]{forest}
\usepackage{listings}
\usepackage{multicol}
\usepackage{tikz}
\usetikzlibrary{arrows.meta}

\begin{document}

\title{Algorytmy i struktury danych}
\author{}
\date{}
\maketitle

\section*{Lista zadań 8}

\subsection*{Zadanie 1}
\begin{enumerate}[label=(\alph*)]
    \item Czym różni się haszowanie łańcuchowe od otwartego? \\[1ex]
          W przypadku haszowania łańcuchowego elementy przechowywane są na liście związanej z danym indeksem tablicy.
          Natomiast w haszowaniu otwartym elementy przechowywane są bezpośrednio w tablicy, a w przypadku kolizji
          są umieszczane w innych komórkach tablicy.
    \item Czym różnią się dwie wersje haszowania otwartego: haszowanie liniowe i haszowanie podwójne? \\[1ex]
          W przypadku haszowania liniowego, gdy wystąpi kolizja, element jest umieszczany w następnej wolnej komórce
          tablicy. Natomiast w przypadku haszowania podwójnego, gdy wystąpi kolizja, element jest umieszczany w komórce
          określonej przez drugą funkcję haszującą.
    \item Dla tablicy z haszowaniem podwójnym o rozmiarze $m=11$ i funkcjach haszujących:
          $h_1(x)=x \mathop{\mathrm{mod}} 11$  oraz $h_2(x)= x \mathop{\mathrm{mod}} 10+1$ wyznacz ciąg kontrolny
          dla liczby 23. Jak wyglądałby ten ciąg w przypadku haszowania liniowego?
          \begin{center}
              \begin{tabular}{c | c}
                  \textbf{haszowanie podwójne}     & \textbf{haszowanie liniowe}      \\
                  \hline
                  1, 5, 9, 2, 6, 10, 3, 7, 0, 4, 8 & 1, 2, 3, 4, 5, 6, 7, 8, 9, 10, 0
              \end{tabular}
          \end{center}
\end{enumerate}

\subsection*{Zadanie 2}
(2pkt) Porównaj jaka będzie łączna liczba kolizji, gdy do tablicy z poprzedniego zadania wstawimy kolejno liczby:
22, 66, 44, 23, 35,  używając: (a) haszowania liniowego, (b) haszowania dwukrotnego, (c) haszowania łańcuchowego. \\[1ex]
\noindent
Następnie w każdym z wariantów sprawdź, jaka będzie łączna liczba porównań kluczy, gdy w gotowej tablicy wywołamy
kolejno procedurę FIND (a) dla każdego elementu obecnego w tablicy, (b) dla elementów: 24 i 34, których nie ma w tablicy.
\begin{center}
    \begin{tabular}{c|c|c|c}
        \textbf{Metoda}       & \textbf{Liczba kolizji} & \textbf{Liczba porównań (a)} & \textbf{Liczba porównań (b)} \\
        \hline
        haszowanie liniowe    & 7                       & 12                           & 7                            \\
        haszowanie podwójne   & 2                       & 7                            & 4                            \\
        haszowanie łańcuchowe & 2                       & 8                            & 2
    \end{tabular}
\end{center}

\pagebreak
\subsection*{Zadanie 3}
Narysuj przykładowe kopce dwumianowe o 12, 14, 15 i 16 węzłach. Na rysunku uwzględnij wartości kluczy oraz stopnie
węzłów. W kopcu 12-elementowym zaznacz dodatkowo strzałki (najlepiej w różnych kolorach) przedstawiające wskaźniki
na ojca, syna i brata.
\begin{center}
    \begin{forest}
        [, phantom, for tree={circle, draw, minimum size=3ex, inner sep=1pt, s sep=5mm, edge=Latex-, calign=last},
            [12[9[6]][11]]{\draw[-Latex] () to node{} (1);}
                [10, name=1[8[2[1]][7]][4[3]][5]]
        ]
    \end{forest}
    \qquad
    \begin{forest}
        [, phantom, for tree={circle, draw, minimum size=3ex, inner sep=1pt, s sep=5mm, edge=Latex-, calign=last},
            [14[13]]{\draw[-Latex] () to node{} (1);}
                [12, name=1[9[6]][11]]{\draw[-Latex] () to node{} (2);}
                [10, name=2[8[2[1]][7]][4[3]][5]]
        ]
    \end{forest}
\end{center}
\begin{center}
    \begin{forest}
        [, phantom, for tree={circle, draw, minimum size=3ex, inner sep=1pt, s sep=5mm, edge=Latex-, calign=last},
            [15]{\draw[-Latex] () to node{} (1);}
                [14, name=1[13]]{\draw[-Latex] () to node{} (2);}
                [12, name=2[9[6]][11]]{\draw[-Latex] () to node{} (3);}
                [10, name=3[8[2[1]][7]][4[3]][5]]
        ]
    \end{forest}
    \qquad
    \begin{forest}
        [, phantom, for tree={circle, draw, minimum size=3ex, inner sep=1pt, s sep=5mm, edge=Latex-, calign=last},
            [16[10[8[2[1]][7]][4[3]][5]][12[9[6]][11]][14[13]][15]]
        ]
    \end{forest}
\end{center}

\subsection*{Zadanie 4}
Zilustruj działanie operacji UNION łączącej kopiec dwumianowy o 14 węzłach z kopcem
dwumianowym o 11 węzłach. Przyjmij dowolne wartości kluczy spełniające warunek kopca.
\begin{center}
    \begin{forest}
        [, phantom, for tree={circle, draw, minimum size=3ex, inner sep=1pt, s sep=5mm, edge=Latex-, calign=last},
            [14[13]]{\draw[-Latex] () to node{} (1);}
                [12, name=1[10[9]][11]]{\draw[-Latex] () to node{} (2);}
                [8, name=2[4[2[1]][3]][6[5]][7]]
        ]
    \end{forest}
    \qquad
    \begin{forest}
        [, phantom, for tree={circle, draw, minimum size=3ex, inner sep=1pt, s sep=5mm, edge=Latex-, calign=last},
            [25]{\draw[-Latex] () to node{} (1);}
                [24, name=1[23]]{\draw[-Latex] () to node{} (2);}
                [22, name=2[18[16[15]][17]][20[19]][21]]
        ]
    \end{forest}
\end{center}
\begin{center}
    \begin{forest}
        [, phantom, for tree={circle, draw, minimum size=3ex, inner sep=1pt, s sep=5mm, edge=Latex-, calign=last},
            [25]{\draw[-Latex] () to node{} (1);}
                [14, name=1[13]]{\draw[-Latex] () to node{} (2);}
                [24, name=2[23]]{\draw[-Latex] () to node{} (3);}
                [12, name=3[10[9]][11]]{\draw[-Latex] () to node{} (4);}
                [8,  name=4[4[2[1]][3]][6[5]][7]]{\draw[-Latex] () to node{} (5);}
                [22, name=5[18[16[15]][17]][20[19]][21]]
        ]
    \end{forest}
\end{center}
\begin{center}
    \begin{forest}
        [, phantom, for tree={circle, draw, minimum size=3ex, inner sep=1pt, s sep=5mm, edge=Latex-, calign=last},
            [25]{\draw[-Latex] () to node{} (1);}
                [24, name=1[14[13]][23]]{\draw[-Latex] () to node{} (2);}
                [12, name=2[10[9]][11]]{\draw[-Latex] () to node{} (3);}
                [8,  name=3[4[2[1]][3]][6[5]][7]]{\draw[-Latex] () to node{} (4);}
                [22, name=4[18[16[15]][17]][20[19]][21]]
        ]
    \end{forest}
\end{center}
\begin{center}
    \begin{forest}
        [, phantom, for tree={circle, draw, minimum size=3ex, inner sep=1pt, s sep=5mm, edge=Latex-, calign=last},
            [25]{\draw[-Latex] () to node{} (1);}
                [24, name=1[12[10[9]][11]][14[13]][23]]{\draw[-Latex] () to node{} (2);}
                [8,  name=2[4[2[1]][3]][6[5]][7]]{\draw[-Latex] () to node{} (3);}
                [22, name=3[18[16[15]][17]][20[19]][21]]
        ]
    \end{forest}
\end{center}
\begin{center}
    \begin{forest}
        [, phantom, for tree={circle, draw, minimum size=3ex, inner sep=1pt, s sep=5mm, edge=Latex-, calign=last},
            [25]{\draw[-Latex] () to node{} (1);}
                [24, name=1[12[10[9]][11]][14[13]][23]]{\draw[-Latex] () to node{} (2);}
                [22, name=2[8[4[2[1]][3]][6[5]][7]][18[16[15]][17]][20[19]][21]]
        ]
    \end{forest}
\end{center}

\subsection*{Zadanie 5}
Udowodnij, że:
\begin{enumerate}[label=(\alph*)]
    \item drzewo dwumianowe rzędu $n$ ma $2^n$ węzłów.
          \begin{gather*}
              B_{k-1} + B_{k-1} = B_k \\
              2^{k-1} + 2^{k-1} = 2 \cdot 2^{k-1} = 2^k
          \end{gather*}
    \item na $k$-tym poziomie drzewa dwumianowego rzędu $n$ znajduje się dokładnie $n \choose k$ węzłów.
          \begin{gather*}
              N(k,i) \implies \text{ilość węzłów w drzewie rzędu k na poziomie i} \\
              N(0,0) \implies {0 \choose 0} = 1 \\
              N(k+1,i) = N(k,i) + N(k,i-1)  \\
              {k+1 \choose i} = {k \choose i} + {k \choose i-1} = \\
              = \frac{k!}{i!(k-i)!} + \frac{k!}{(i-1)!(k-i+1)!} = \frac{k!(k-i+1)}{i!(k-i+1)!} + \frac{k!i}{i!(k-i+1)!} = \\
              \frac{k!(k-i+1+i)}{i!(k-i+1)!} = \frac{(k+1)!}{i!(k+1-i)!} = {k+1 \choose i}
          \end{gather*}
\end{enumerate}

\subsection*{Zadanie 6}
Napisz funkcję \verb`int ile_drzew_w_kopcu(int n)` wyliczającą ile jest drzew dwumianowych w kopcu
dwumianowym zawierającym $n$ kluczy.
\begin{lstlisting}
                            int ile_drzew_w_kopcu(int n)
                            {
                                int i = 0;
                                while (n > 0)
                                {
                                    if (n % 2 == 1) ++i;
                                    n /= 2;
                                }
                                return i;
                            }
\end{lstlisting}

\subsection*{Zadanie 7}
\begin{enumerate}[label=(\alph*)]
    \item Do pustego kopca dwumianowego wstaw (INSERT) kolejno: 1, 12, 3, 14, 5, 16, 7, 20, 25, 13, 8
          \begin{center}
              \begin{forest}
                  [, phantom, for tree={circle, draw, minimum size=3ex, inner sep=1pt, s sep=5mm, edge=Latex-, calign=last},
                      [8]{\draw[-Latex] () to node{} (1);}
                          [25, name=1[13]]{\draw[-Latex] () to node{} (2);}
                          [20, name=2[14[12[1]][3]][16[5]][7]]
                  ]
              \end{forest}
          \end{center}
    \item Dla otrzymanego kopca dwukrotnie wykonaj operację GETMAX.
          \begin{center}
              \begin{forest}
                  [, phantom, for tree={circle, draw, minimum size=3ex, inner sep=1pt, s sep=5mm, edge=Latex-, calign=last},
                      [13[8]]{\draw[-Latex] () to node{} (1);}
                          [20, name=1[14[12[1]][3]][16[5]][7]]
                  ]
              \end{forest}
              \qquad
              \begin{forest}
                  [, phantom, for tree={circle, draw, minimum size=3ex, inner sep=1pt, s sep=5mm, edge=Latex-, calign=last},
                      [7]{\draw[-Latex] () to node{} (1);}
                          [16, name=1[14[12[1]][3]][13[8]][5]]
                  ]
              \end{forest}
          \end{center}
\end{enumerate}

\end{document}
