\documentclass{article}

\usepackage{amsmath}
\usepackage{amssymb}
\usepackage{enumitem}
\usepackage[polish]{babel}
\usepackage[T1]{fontenc}
\usepackage[utf8]{inputenc}
\usepackage[margin=0.8in]{geometry}
\usepackage[edges]{forest}
\usepackage{listings}
\usepackage{multicol}
\usepackage{varwidth}
\usepackage{xcolor}

\begin{document}

\title{Algorytmy i struktury danych}
\author{}
\date{}
\maketitle

Dana jest struktura danych będąca węzłem drzewa BST
\begin{lstlisting}
    struct node {
        int key;
        node* left;
        node* right;
        node(int k, node* l, node* r) : key(k), left(l), right(r) {}
    };
\end{lstlisting}

\subsection*{Zadanie 1}
Zapisz warunki jakie muszą spełniać klucze drzewa BST. \\
\noindent
Klucze w lewym poddrzewie są mniejsze od klucza węzła, natomiast w prawym poddrzewie są większe lub równe.

\subsection*{Zadanie 2}
Napisz procedurę \verb|node* find(node* tree, int x)|, która zwraca wskaźnik na węzeł
zawierający x, lub NULL, jeśli nie ma takiego węzła.
\begin{lstlisting}
    node* find(node* tree, int x) {
        if (tree == nullptr) return nullptr;
        if (tree->key == x) return tree;
        if (tree->key > x) return find(tree->left, x);
        return find(tree->right, x);
    }
\end{lstlisting}

\subsection*{Zadanie 3}
Napisz procedurę \verb|void insert(node*& tree, int x)| (dodaje do drzewa tree klucz x).
\begin{lstlisting}
    void insert(node*& tree, int x) {
        if (tree == nullptr) {
            tree = new node(x, nullptr, nullptr);
            return;
        }
        if (tree->key > x) insert(tree->left, x);
        else insert(tree->right, x);
    }
\end{lstlisting}

\pagebreak
\subsection*{Zadanie 4}
Drzewo BST o różnych kluczach można odtworzyć z listy par kluczWezła:kluczOjca. (a) Narysuj
drzewo BST reprezentowane przez listę par: 1:2, 2:4, 3:2, 4:5, 6:7, 7:9, 8:7, 9:5.
(b) wypisz jego klucze w porządku: INORDER, (c) PREORDER, (d) POSTORDER
\begin{center}
    \begin{forest}
        for tree={circle, draw, minimum size=3ex, inner sep=1pt, s sep=5mm, anchor=south},
        [5[4[2[1][3]][, phantom]][9[7][, phantom]]]
    \end{forest}
\end{center}
\begin{enumerate}[label=(\alph*)]
    \setcounter{enumi}{1}
    \item 1234567
    \item 5421397
    \item 1324795
\end{enumerate}

\subsection*{Zadanie 5}
Napisz procedurę \verb|void wypisz(node *tree, int order=0)|, która wypisuje klucze drzewa tree
w porządku inorder gdy order=0, preorder gdy order=1, postorder gdy order=2.
\begin{lstlisting}
    void wypisz(node *tree, int order = 0) {
        if (tree == nullptr) return;
        if (order == 1) std::cout << tree->key;
        wypisz(tree->left, order);
        if (order == 0) std::cout << tree->key;
        wypisz(tree->right, order);
        if (order == 2) std::cout << tree->key;
    }
\end{lstlisting}

\subsection*{Zadanie 6}
Jakie informacje przechowujemy w węźle drzewa czerwono-czarnego? Podaj definicję drzewa
czerwono czarnego. Zadeklaruj strukturę RBnode tak, by dziedziczyła z node. Czy można dla niej użyć funkcji napisanych
w zadaniach 2, 3 i 5?
\begin{lstlisting}
    struct NodeBST
    {
        int32_t value;
        NodeBST* left;
        NodeBST* right;  
    };

    struct NodeRBT : public NodeBST
    {
        bool isBlack;
        NodeRBT* parent;
    };
\end{lstlisting}
Drzewo czerwono-czarne musi przestrzegać następujące wymagania:
\begin{enumerate}
    \item Każdy węzeł jest albo czerwony albo czarny.
    \item Korzeń jest czarny.
    \item Każdy liść jest czarny.
    \item Czerwony węzeł ma czarnego ojca.
    \item Każda ścieżka od korzenia do liścia ma tę samą liczbę czarnych węzłów.
\end{enumerate}
Można użyć dla niej funkcji \verb|find| i \verb|wypisz| z zadań 2 i 5, ale nie \verb|insert| z zadania 3.

\subsection*{Zadanie 7}
Uzasadnij posługując się rysunkiem i opisem, że operacje na drzewie czerwono-czarnym (rotacja i
przekolorowanie) nie zmieniają ilości czarnych węzłów, na żadnej ścieżce od korzenia do liścia
\begin{multicols}{2}
    \begin{center}
        \begin{forest}
            [, phantom, for tree={circle, minimum size=3ex, inner sep=1pt, s sep=5mm, anchor=south, fill=black, text=white},
            [1[a, fill=none, text=black][2, fill=red[b, fill=none, text=black][c, fill=none, text=black]]]
            [2[1, fill=red[a, fill=none, text=black][b, fill=none, text=black]][c, fill=none, text=black]]
            ]
        \end{forest} \\
        Po wykonaniu rotacji liczba czarnych węzłów na ścieżce nie ulega zmianie, a rotowane węzły wymieniają się
        piętrami i kolorami.
    \end{center}
    \columnbreak
    \begin{center}
        \begin{forest}
            [, phantom, for tree={circle, minimum size=3ex, inner sep=1pt, s sep=5mm, anchor=south, fill=none, text=black},
            [a[2, fill=black, text=white[1, fill=red, text=white[a][b]][3, fill=red, text=white[d][e]]][, phantom]]
            [a[2, fill=red, text=white[1, fill=black, text=white[a][b]][3, fill=black, text=white[d][e]]][, phantom]]
            ]
        \end{forest} \\
        Kolory czarne z ścieżek wychodzących zostają wypchnięte do węzła nadrzędnego.
    \end{center}
\end{multicols}

\subsection*{Zadanie 8}
W poniższym drzewie czerwono-czarnym (czarne węzły oznaczono nawiasem kwadratowym), usuń 1,
dodaj do wyściowego 10:
\begin{center}
    \begin{forest}
        for tree={circle, minimum size=3ex, inner sep=1pt, s sep=7mm, anchor=south, fill=black, text=white},
        [5[3, fill=red[1][4]][11[9, fill=red][, phantom]]]
    \end{forest}
\end{center}
\begin{center}
    \begin{forest}
        [, phantom, for tree={circle, minimum size=3ex, inner sep=1pt, s sep=5mm, anchor=south, fill=black, text=white},
        [5[3, fill=red[n][4]][11[9, fill=red][, phantom]]]
        [5[3[, phantom][4, fill=red]][11[9, fill=red][, phantom]]]
        ]
    \end{forest}
\end{center}
\begin{center}
    \begin{forest}
        [, phantom, for tree={circle, minimum size=3ex, inner sep=1pt, s sep=5mm, anchor=south, fill=black, text=white},
        [5[3[, phantom][4, fill=red]][11[9, fill=red[, phantom][10, fill=red]][, phantom]]]
        [5[3[, phantom][4, fill=red]][11[10, fill=red[9, fill=red][, phantom]][, phantom]]]
        [5[3[, phantom][4, fill=red]][10[9, fill=red][11, fill=red]]]
        ]
    \end{forest}
\end{center}

\subsection*{Zadanie 9}
Jakie informacje przechowujemy w węźle B-drzewa? Podaj definicję B-drzewa
\begin{lstlisting}
    struct BTree
    {
        uint32_t t;
        bool isLeaf;
        size_t n;
        int32_t *keys;
        size_t *offsets;
    };
\end{lstlisting}
\begin{enumerate}
    \item Każdy węzeł posiada $n$ kluczy, przechowywanych w kolejności niemalejącej, a także informację o tym czy jest
          on liściem.
    \item Dodatkowo każdy węzeł posiada $n+1$ wskaźników do swoich dzieci.
    \item Klucze węzła dzielą zbiór kluczy przechowywanych w jego dzieciach na $n+1$ przedziałów.
    \item Wszystkie liście znajdują się na tym samym poziomie równym wysokości drzewa $h$.
    \item Każdy węzeł, z wyjątkiem korzenia, posiada co najmniej $t-1$ kluczy.
    \item Każdy węzeł może posiadać maksymalnie $2t-1$ kluczy.
\end{enumerate}

\subsection*{Zadanie 10}
Narysuj B-drzewo o t = 3 zawierające dokładnie 17 kluczy na trzech poziomach: korzeń jego
dzieci i wnuki. Następnie usuń z tego drzewa korzeń.
\tikzstyle{every picture}+=[baseline=(current bounding box.north)]
\tikzstyle{bplus}=[rectangle split, rectangle split horizontal, rectangle split ignore empty parts, draw]
\tikzstyle{every node}=[bplus]
\tikzstyle{every path}+=[level distance=15mm]
\tikzstyle{level 1}=[sibling distance=60mm]
\tikzstyle{level 2}=[sibling distance=20mm]
\begin{center}
    \begin{tikzpicture}
        \node {9} [->]
        child {node {3 \nodepart{two} 6}
                child {node {1 \nodepart{two} 2}}
                child {node {4 \nodepart{two} 5}}
                child {node {7 \nodepart{two} 8}}
            }
        child {node {12 \nodepart{two} 15}
                child {node {10 \nodepart{two} 11}}
                child {node {13 \nodepart{two} 14}}
                child {node {16 \nodepart{two} 17}}
            };
    \end{tikzpicture}
\end{center}
\tikzstyle{level 1}=[sibling distance=20mm]
\begin{center}
    \begin{tikzpicture}
        \node [rectangle split parts=5] {3 \nodepart{two} 6 \nodepart{three} 9 \nodepart{four} 12 \nodepart{five} 15} [->]
        child {node {1 \nodepart{two} 2}}
        child {node {4 \nodepart{two} 5}}
        child {node {7 \nodepart{two} 8}}
        child {node {10 \nodepart{two} 11}}
        child {node {13 \nodepart{two} 14}}
        child {node {16 \nodepart{two} 17}};
    \end{tikzpicture}
\end{center}
\begin{center}
    \begin{tikzpicture}
        \node [rectangle split parts=5] {3 \nodepart{two} 6 \nodepart{three} 12 \nodepart{four} 15} [->]
        child {node {1 \nodepart{two} 2}}
        child {node {4 \nodepart{two} 5}}
        child {node {7 \nodepart{two} 8 \nodepart{three} 10 \nodepart{four} 11}}
        child {node {13 \nodepart{two} 14}}
        child {node {16 \nodepart{two} 17}};
    \end{tikzpicture}
\end{center}

\subsection*{Zadanie 11}
Podano na rysunku B-drzewo o $t=2$:
\tikzstyle{level 1}=[sibling distance=60mm]
\tikzstyle{level 2}=[sibling distance=20mm]
\begin{center}
    \begin{tikzpicture}
        \node {9} [->]
        child {node {7}
                child {node {6}}
                child {node {8}}
            }
        child {node {11 \nodepart{two} 14 \nodepart{three} 19}
                child {node {10}}
                child {node {12 \nodepart{two} 13}}
                child {node {15 \nodepart{two} 16 \nodepart{three} 17}}
                child {node {20}}
            };
    \end{tikzpicture}
    \begin{enumerate}[label=-]
        \item usuń z tego drzewa 7.
    \end{enumerate}
    \begin{tikzpicture}
        \node {11} [->]
        child {node {9}
                child {node {6 \nodepart{two} 8}}
                child {node {10}}
            }
        child {node {14 \nodepart{two} 19}
                child {node {12 \nodepart{two} 13}}
                child {node {15 \nodepart{two} 16 \nodepart{three} 17}}
                child {node {20}}
            };
    \end{tikzpicture}
    \begin{enumerate}[label=-]
        \item do drzewa widocznego powyżej dodaj 18.
    \end{enumerate}
    \begin{tikzpicture}
        \node {9 \nodepart{two} 14} [->]
        child {node {7}
                child {node {6}}
                child {node {8}}
            }
        child {node {11}
                child {node {10}}
                child {node {12 \nodepart{two} 13}}
            }
        child {node {19}
                child {node {15 \nodepart{two} 16 \nodepart{three} 17}}
                child {node {20}}
            };
    \end{tikzpicture}\\[1em]
    \begin{tikzpicture}
        \node {9 \nodepart{two} 14} [->]
        child {node {7}
                child {node {6}}
                child {node {8}}
            }
        child {node {11}
                child {node {10}}
                child {node {12 \nodepart{two} 13}}
            }
        child {node {16 \nodepart{two} 19}
                child {node {15}}
                child {node {17 \nodepart{two} 18}}
                child {node {20}}
            };
    \end{tikzpicture}
\end{center}

\subsection*{Zadanie 12}
W B-drzewie o t = 10:
\begin{enumerate}[label=(\alph*)]
    \item ile kluczy może zawierać korzeń (podaj przedział), \\
          Korzeń zawiera od 1 do 19 kluczy. (max $2t-1$)
    \item ile dzieci może mieć korzeń (podaj przedział), \\
          Korzeń może mieć od 2 do 20 dzieci. (min $t$ max $2t$)
    \item ile kluczy może mieć potomek korzenia (podaj przedział), \\
          Potomek korzenia może mieć od 9 do 19 kluczy. (min $t-1$ max $2t-1$)
    \item ile dzieci może mieć potomek korzenia (podaj przedział), \\
          Potomek korzenia może mieć od 10 do 20 dzieci. (min $t$ max $2t$)
    \item ile maksymalnie węzłów może być na $k$-tym poziomie (przyjmując, że korzeń to poziom $0$) \\
          Na k-tym poziomie może być maksymalnie $(2t)^k$ węzłów.
    \item ile łącznie kluczy może być na $k$-tym poziomie (podaj przedział). \\
          Nie licząc korzenia dla którego minimum to 1 klicz to na k-tym poziomie może być od $2(t-1)t^{k-1}$ do $(2t-1)(2t)^{k}$ kluczy. (min $(2min)t^{k-1}$ max $(max)t^{k}$
\end{enumerate}

\end{document}