\documentclass{article}

\usepackage{amsmath}
\usepackage{amssymb}
\usepackage{enumitem}
\usepackage[polish]{babel}
\usepackage[T1]{fontenc}
\usepackage[utf8]{inputenc}
\usepackage[margin=0.8in]{geometry}
\usepackage[edges]{forest}
\usepackage{listings}
\usepackage{multicol}
\usepackage{varwidth}
\usepackage{xcolor}

\begin{document}

\title{Algorytmy i struktury danych}
\author{}
\date{}
\maketitle

Na podstawie grafu nieskierowanego zadanego jako następująca lista krawędzi:
\begin{center}
    (1,2):8 (2,3):1 (3,4):15 (2,5):7 (4,5):3 (5,6):12 (1,6):2 (6,7):4 (2,6):20 (5,7):5
\end{center}

\subsection*{Zadanie 1}
Wykonaj rysunek grafu.

\subsection*{Zadanie 2}
Znajdź macierz sąsiedztwa.

\subsection*{Zadanie 3}
3. Zapisz tablicę list sąsiedztwa. Wierzchołki na listach sąsiedztwa powinny być są ustawione
rosnąco wg numeru wierzchołka. Ta kolejność powinna być stosowana w symulacji algoryt-
mów DFS, BFS i Dijkstry.

\subsection*{Zadanie 4}
Zapisz kolejność odwiedzania wierzchołków przez algorytm DFS startujący z wierzchołka 5.

\subsection*{Zadanie 2}
Zapisz kolejność odwiedzania wierzchołków w algorytmie BFS startującym z tego samego
wierzchołka.

\subsection*{Zadanie 2}
(2 pkt) Zasymuluj działanie algorytmu Kruskala i zilustruj rysunkiem:
\begin{itemize}
    \item liniami przerywanymi oznacz krawędzie nie należące do drzewa wynikowego,
    \item liniami ciągłymi oznacz krawędzie należące do drzewa wynikowego,
    \item przy każdej krawędzi w nawiasie okrągłym podaj kolejność w jakiej była ona rozpatrywana.
\end{itemize}

\subsection*{Zadanie 2}
(3 pkt) Zasymuluj działanie algorytmu Dijksty startując z wierchołka 3. Zapisz kroki algo-
rytmu podając w każdym kroku:
\begin{itemize}
    \item numer odwiedzanego wierzchołka
    \item wykonane w tym kroku operacje \verb|decrease_key| i odpowiednie zmiany w tablicy poprzedników (\verb|prev|)
    \item wypisując jaka jest zawartość kolejki priorytetowej po wykonaniu kroku
\end{itemize}
Na końcu algorytmu dla każdego wierzchołka zapisz:
\begin{itemize}
    \item odległość od wierzchołka startowego
    \item numer wierzchołka będącego poprzednikiem
\end{itemize}
Algorytm zilustruj grafem, w którym:
\begin{itemize}
    \item przy każdym wierzchołku będzie podany w nawiasie okrągłym numer kroku algorytmu,
          w którym wierzchołek został odwiedzony.
    \item strzałkami ciągłymi oznaczone będą krawędzie należące do drzewa wynikowego
    \item strzałkami przerywanymi oznaczone będą krawędzie, które w trakcie algorytmu wskazywały
          na poprzednika, jednak nie należą do drzewa wynikowego.
\end{itemize}

\end{document}