\documentclass{article}

\usepackage{amsmath}
\usepackage{amssymb}
\usepackage{enumitem}
\usepackage[polish]{babel}
\usepackage[T1]{fontenc}
\usepackage[utf8]{inputenc}
\usepackage[margin=0.8in]{geometry}
\usepackage[edges]{forest}
\usepackage{listings}
\usepackage{multicol}
\usepackage{multirow}
\usepackage{varwidth}
\usepackage{xcolor}
\usepackage{tikz}
\usetikzlibrary{arrows.meta}
\tikzset{
    v/.style={draw, circle, node distance={20mm}, thick}, 
    e/.style={midway, sloped, above}}

\begin{document}

\title{Algorytmy i struktury danych}
\author{}
\date{}
\maketitle

Na podstawie grafu nieskierowanego zadanego jako następująca lista krawędzi:
\begin{center}
    (1,2):8 (2,3):1 (3,4):15 (2,5):7 (4,5):3 (5,6):12 (1,6):2 (6,7):4 (2,6):20 (5,7):5
\end{center}

\subsection*{Zadanie 1}
Wykonaj rysunek grafu.
\begin{center}
    \begin{tikzpicture}
        \node[v] (1) {1};
        \node[v] (2) [right of=1] {2};
        \node[v] (3) [right of=2] {3};
        \node[v] (4) [below of=3] {4};
        \node[v] (5) [below of=2] {5};
        \node[v] (6) [left of=5] {6};
        \node[v] (7) [below of=6] {7};
        \draw (1) -- node[e] {8} (2);
        \draw (2) -- node[e] {1} (3);
        \draw (3) -- node[e] {15} (4);
        \draw (2) -- node[e] {7} (5);
        \draw (4) -- node[e] {3} (5);
        \draw (5) -- node[e] {12} (6);
        \draw (1) -- node[e] {2} (6);
        \draw (6) -- node[e] {4} (7);
        \draw (2) -- node[e] {20} (6);
        \draw (5) -- node[e] {5} (7);
    \end{tikzpicture}
\end{center}

\subsection*{Zadanie 2}
Znajdź macierz sąsiedztwa.
\begin{center}$
        \begin{pmatrix}
              & 1 & 2  & 3  & 4  & 5  & 6  & 7 \\
            1 & 0 & 8  & 0  & 0  & 0  & 2  & 0 \\
            2 & 8 & 0  & 1  & 0  & 7  & 20 & 0 \\
            3 & 0 & 1  & 0  & 15 & 0  & 0  & 0 \\
            4 & 0 & 0  & 15 & 0  & 3  & 0  & 0 \\
            5 & 0 & 7  & 0  & 3  & 0  & 12 & 5 \\
            6 & 2 & 20 & 0  & 0  & 12 & 0  & 4 \\
            7 & 0 & 0  & 0  & 0  & 5  & 4  & 0
        \end{pmatrix}$
\end{center}

\subsection*{Zadanie 3}
Zapisz tablicę list sąsiedztwa. Wierzchołki na listach sąsiedztwa powinny być są ustawione
rosnąco wg numeru wierzchołka. Ta kolejność powinna być stosowana w symulacji algorytmów DFS, BFS i Dijkstry.
\begin{center}
    \begin{tabular}{|c|c|}
        \hline
        Wierzchołek & Lista sąsiedztwa \\
        \hline
        1           & 2, 6             \\
        \hline
        2           & 1, 3, 5, 6       \\
        \hline
        3           & 2, 4             \\
        \hline
        4           & 3, 5             \\
        \hline
        5           & 2, 4, 6, 7       \\
        \hline
        6           & 1, 2, 5, 7       \\
        \hline
        7           & 5, 6             \\
        \hline
    \end{tabular}
\end{center}

\pagebreak
\subsection*{Zadanie 4}
Zapisz kolejność odwiedzania wierzchołków przez algorytm DFS startujący z wierzchołka 5.
\begin{center}
    5, 2, 1, 6, 7, 3, 4
\end{center}

\subsection*{Zadanie 5}
Zapisz kolejność odwiedzania wierzchołków w algorytmie BFS startującym z tego samego
wierzchołka.
\begin{center}
    5, 2, 4, 6, 7, 1, 3
\end{center}

\subsection*{Zadanie 6}
(2 pkt) Zasymuluj działanie algorytmu Kruskala i zilustruj rysunkiem:
\begin{itemize}
    \item liniami przerywanymi oznacz krawędzie nie należące do drzewa wynikowego,
    \item liniami ciągłymi oznacz krawędzie należące do drzewa wynikowego,
    \item przy każdej krawędzi w nawiasie okrągłym podaj kolejność w jakiej była ona rozpatrywana.
\end{itemize}
\tikzset{v/.append style={node distance=30mm}}
\begin{center}
    \begin{tikzpicture}
        \node[v] (1) {1};
        \node[v] (2) [right of=1] {2};
        \node[v] (3) [right of=2] {3};
        \node[v] (4) [below of=3] {4};
        \node[v] (5) [below of=2] {5};
        \node[v] (6) [left of=5] {6};
        \node[v] (7) [below of=6] {7};
        \draw (2) -- node[e] {1} node[e, below] {(1)} (3);
        \draw (1) -- node[e] {2} node[e, below] {(2)} (6);
        \draw (4) -- node[e] {3} node[e, below] {(3)} (5);
        \draw (6) -- node[e] {4} node[e, below] {(4)} (7);
        \draw (5) -- node[e] {5} node[e, below] {(5)} (7);
        \draw (2) -- node[e] {7} node[e, below] {(6)} (5);
        \draw[dashed] (1) -- node[e] {8} node[e, below] {(7)} (2);
        \draw[dashed] (5) -- node[e] {12} node[e, below] {(8)} (6);
        \draw[dashed] (3) -- node[e] {15} node[e, below] {(9)} (4);
        \draw[dashed] (2) -- node[e] {20} node[e, below] {(10)} (6);
    \end{tikzpicture}
\end{center}

\subsection*{Zadanie 7}
(3 pkt) Zasymuluj działanie algorytmu Dijkstry startując z wierchołka 3. Zapisz kroki algorytmu podając w każdym kroku:
\begin{itemize}
    \item numer odwiedzanego wierzchołka
    \item wykonane w tym kroku operacje \verb|decrease_key| i odpowiednie zmiany w tablicy poprzedników (\verb|prev|)
    \item wypisując jaka jest zawartość kolejki priorytetowej po wykonaniu kroku
\end{itemize}
\begin{center}
    \begin{tabular}{*{5}{|c}|}
        \hline
        \textbf{Wierzchołek} & \multicolumn{3}{|c|}{\textbf{Operacje}} & \textbf{Kolejka}                                                                            \\
        \hline
        -                    & \verb|decrease_key(3, 0)|               & \verb|prev[3] = 3|         &                            & 3, 1, 2, 4, 5, 6, 7               \\
        \hline
        \multirow{2}{*}{3}   & \verb|decrease_key(2, 1)|               & \verb|decrease_key(4, 15)| &                            & \multirow{2}{*}{2, 4, 1, 5, 6, 7} \\
                             & \verb|prev[2] = 3|                      & \verb|prev[4] = 3|         &                            &                                   \\
        \hline
        \multirow{2}{*}{2}   & \verb|decrease_key(1, 9)|               & \verb|decrease_key(5, 8)|  & \verb|decrease_key(6, 21)| & \multirow{2}{*}{5, 1, 4, 6, 7}    \\
                             & \verb|prev[1] = 2|                      & \verb|prev[5] = 2|         & \verb|prev[6] = 2|         &                                   \\
        \hline
        \multirow{2}{*}{5}   & \verb|decrease_key(4, 11)|              & \verb|decrease_key(6, 20)| & \verb|decrease_key(7, 13)| & \multirow{2}{*}{1, 4, 7, 6}       \\
                             & \verb|prev[4] = 5|                      & \verb|prev[6] = 5|         & \verb|prev[7] = 5|         &                                   \\
        \hline
        1                    & \verb|decrease_key(6, 11)|              & \verb|prev[6] = 1|         &                            & 4, 6, 7                           \\
        \hline
        4                    &                                         &                            &                            & 6, 7                              \\
        \hline
        6                    &                                         &                            &                            & 7                                 \\
        \hline
        7                    &                                         &                            &                            & -                                 \\
        \hline
    \end{tabular}
\end{center}
Na końcu algorytmu dla każdego wierzchołka zapisz:
\begin{itemize}
    \item odległość od wierzchołka startowego
    \item numer wierzchołka będącego poprzednikiem
\end{itemize}
\begin{center}
    \begin{tabular}{*{8}{|c}|}
        \hline
        \textbf{Wierzchołek} & 1 & 2 & 3 & 4  & 5 & 6  & 7  \\
        \hline
        \textbf{Odległość}   & 9 & 1 & 0 & 11 & 8 & 11 & 13 \\
        \hline
        \textbf{Poprzednik}  & 2 & 3 & 3 & 5  & 2 & 1  & 5  \\
        \hline
    \end{tabular}
\end{center}
Algorytm zilustruj grafem, w którym:
\begin{itemize}
    \item przy każdym wierzchołku będzie podany w nawiasie okrągłym numer kroku algorytmu,
          w którym wierzchołek został odwiedzony.
    \item strzałkami ciągłymi oznaczone będą krawędzie należące do drzewa wynikowego
    \item strzałkami przerywanymi oznaczone będą krawędzie, które w trakcie algorytmu wskazywały
          na poprzednika, jednak nie należą do drzewa wynikowego.
\end{itemize}
\begin{center}
    \begin{tikzpicture}
        \node[v, label=(4)] (1) {1};
        \node[v, label=(2)] (2) [right of=1] {2};
        \node[v, label=(1)] (3) [right of=2] {3};
        \node[v, label=0:(5)] (4) [below of=3] {4};
        \node[v, label=-45:(3)] (5) [below of=2] {5};
        \node[v, label=180:(6)] (6) [left of=5] {6};
        \node[v, label=-90:(7)] (7) [below of=6] {7};
        \draw[-Latex] (2) -- node[e] {1} (3);
        \draw[Latex-] (1) -- node[e] {2} (6);
        \draw[-Latex] (4) -- node[e] {3} (5);
        \draw[dashed] (6) -- node[e] {4} (7);
        \draw[Latex-] (5) -- node[e] {5} (7);
        \draw[Latex-] (2) -- node[e] {7} (5);
        \draw[-Latex] (1) -- node[e] {8} (2);
        \draw[dashed, Latex-] (5) -- node[e] {12} (6);
        \draw[dashed, Latex-] (3) -- node[e] {15} (4);
        \draw[dashed, Latex-] (2) -- node[e] {20} (6);
    \end{tikzpicture}
\end{center}


\end{document}